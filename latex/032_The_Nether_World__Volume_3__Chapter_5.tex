\chapter{A Haven}

\textsc{John Hewett} no longer had membership in club or society. The
loss of his insurance-money made him for the future regard all such
institutions with angry suspicion. ``Workin' men ain't satisfied with
bein' robbed by the upper classes; they must go and rob one another.''
He had said good-bye to Clerkenwell Green; the lounging crowd no longer
found amusement in listening to his frenzied voice and in watching the
contortions of his rugged features. He discussed the old subjects with
Eagles, but the latter's computative mind was out of sympathy with zeal
of the tumid description; though quite capable of working himself into
madness on the details of the Budget, John was easily soothed by his
friend's calmer habits of debate. Kirkwood's influence, moreover, was
again exerting itself {}upon him,---an influence less than ever likely
to encourage violence of thought or speech. In Sidney's company the worn
rebel became almost placid; his rude, fretted face fell into a singular
humility and mildness. Having ended by accepting what he would formerly
have called charity, and that from a man whom he had wronged with
obstinate perverseness, John neither committed the error of obtruding
his gratitude, nor yet suffered it to be imagined that obligation sat
upon him too lightly. He put no faith in Sidney's assertion that some
unknown benefactor was to be thanked for the new furniture; one and the
same pocket had supplied that and the money for Mrs. Hewett's burial.
Gratitude was all very well, but he could not have rested without taking
some measures towards a literal repayment of his debt. The weekly
coppers which had previously gone for club subscriptions were now put
away in a money-box; they would be long enough in making an appreciable
sum, but yet, if he himself could never discharge the obligation, his
children must take it up after him, and this he frequently impressed
upon Amy, Annie, and Tom.

{}Nothing, however, could have detached John's mind so completely from
its habits of tumult, nor have fixed it so firmly upon the interests of
home, as his recovery of his daughter. From the day of Clara's
establishment under his roof he thought of her, and of her only. Whilst
working at the filter-factory he remained in imagination by her side,
ceaselessly repeating her words of the night before, eagerly looking for
the hour that would allow him to return to her. Joy and trouble mingled
in an indescribable way to constitute his ordinary mood; one moment he
would laugh at a thought, and before a companion could glance at him his
gladness would be overshadowed as if with the heaviest anxiety. Men who
saw him day after day said at this time that he seemed to be growing
childish; he muttered to himself a good deal, and looked blankly at you
when you addressed him. In the course of a fortnight his state became
more settled, but it was not the cheerful impulse that predominated. Out
of the multude of thoughts concerning Clara, one had fixed itself as the
main controller of his reflection. Characteristically, John hit upon
{}what seemed an irremediable misfortune, and brooded over it with all
his might. If only Sidney Kirkwood were in the same mind as four years
ago!

And now was he to believe that what he had been told about Sidney and
Jane Snowdon was misleading? Was the impossible no longer so? He almost
leapt from his chair when he heard that Sidney was the visitor with whom
his daughter had been having her private conversation. How came they to
make this appointment? There was something in Clara's voice that set his
nerves a-tremble. That night he could not sleep, and next mornins; he
went to work with a senile quiver in his body. For the first time for
more than two months he turned into a public-house on his way, just to
give himself a little ``tone.'' The natural result of such a tonic was
to heighten the fever of his imagination; goodness knows how far he had
got in a drama of happiness before he threw off his coat and settled to
his day's labour.

Clara, in the meanwhile, suffered a corresponding agitation, more
penetrative in proportion to the finer substance of her nature. {}She
did not know until the scene was over how much vital force it had cost
her; when she took off the veil a fire danced before her eyes, and her
limbs ached and trembled as she lay down in the darkness. All night long
she was acting her part over and over; when she woke up, it was always
at the point where Sidney replied to her, ``But you are mistaken!''

Acting her part; yes, but a few hours had turned the make-believe into
something earnest enough. She could not now have met Kirkwood with the
self-possession of last evening. The fever that then sustained her was
much the same as she used to know before she had thoroughly accustomed
herself to appearing in front of an audience; it exalted all her
faculties, gifted her with a remarkable self-consciousness. It was all
very well as long as there was need of it, but why did it afflict her in
this torturing form now that she desired to rest, to think of what she
had gained, of what hope she might reasonably nourish? The purely
selfish project which, in her desperation, had seemed the only resource
remaining to her against a life of intolerable desolateness, was taking
hold upon her in a way she could not {}understand. Had she not already
made a discovery that surpassed all expectation? Sidney Kirkwood was not
bound to another woman; why could she not accept that as so much clear
gain, and deliberate as to her next step? She had been fully prepared
for the opposite state of things, prepared to strive against any odds,
to defy all probabilities, all restraints; why not thank her fortune and
plot collectedly now that the chances were so much improved?

But from the beginning of her interview with him, Clara knew that
something more entered into her designs on Sidney than a cold
self-interest. She had never loved him; she never loved any one; yet the
inclinations of her early girlhood had been drawn by the force of the
love he offered her, and to this day she thought of him with a respect
and liking such as she had for no other man. When she heard from her
father that Sidney had forgotten her, had found some one by whom his
love was prized, her instant emotion was so like a pang of jealousy that
she marvelled at it. Suppose fate had prospered her, and she had heard
in the midst of triumphs that Sidney Kirkwood, the working man in
Clerkenwell, was going to {}marry a girl he loved, would any feeling of
this kind have come to her? Her indifference would have been complete.
It was calamity that made her so sensitive. Self-pity longs for the
compassion of others. That Sidney, who was once her slave, should stand
aloof in freedom now that she wanted sympathy so sorely, this was a
wound to her heart. That other woman had robbed her of something she
could not spare.

Jane Snowdon, too! She found it scarcely conceivable that the wretched
little starveling of Mrs. Peckover's kitchen should have grown into
anything that a man like Sidney could love. To be sure, there was a
mystery in her lot. Clara remembered perfectly how Scawthorne pointed
out of the cab at the old man Snowdon, and said that he was very rich. A
miser, or what? More she had never tried to discover. Now Sidney himself
had hinted at something in Jane's circumstances which, he professed, put
it out of the question that he could contemplate marrying her. Had he
told her the truth? Could she in fact consider him free? Might there not
be some reason for his wishing to keep a secret?

{}With burning temples, with feverish lips, she moved about her little
room like an animal in a cage, finding the length of the day
intolerable. She was constrained to inaction, when it seemed to her that
every moment in which she did not do something to keep Sidney in mind of
her was worse than lost. Could she not see that girl, Jane Snowdon? But
was not Sidneys denial as emphatic as it could be? She recalled his
words, and tried numberless interpretations. Would anything that he had
said bear being interpreted as a sign that something; of the old
tenderness still lived in him? And the strange thing was, that she
interrogated herself on these points not at all like a coldly scheming
woman, who aims at something that is to be won, if at all, by the
subtlest practising on another's emotions, whilst she remains
unaffected. Rather like a woman who loves passionately, whose ardour and
jealous dread wax moment by moment.

For what was she scheming? For food, clothing, assured comfort during
her life? Twenty-four hours ago Clara would most likely have believed
that she had indeed fallen to this; but the meeting with Sidney
{}enlightened her. Least of all women could \emph{she} live by bread
alone; there was the hunger of her brain, the hunger of her heart. I
spoke once, you remember, of her ``defect of tenderness;'' the fault
remained, but her heart was no longer so sterile of the tender emotions
as when revolt and ambition absorbed all her energies. She had begun to
feel gently towards her father; it was an intimation of the need which
would presently bring all the forces of her nature into play. She
dreaded a life of drudgery; she dreaded humiliation among her inferiors;
but that which she feared most of all was the barrenness of a lot into
which would enter none of the passionate joys of existence. Speak to
Clara of renunciation, of saintly glories, of the stony way of
perfectness, and you addressed her in an unknown tongue; nothing in her
responded to these ideas. Hopelessly defeated in the one way of
aspiration which promised a large life, her being, rebellious against
the martyrdom it had suffered, went forth eagerly towards the only
happiness which was any longer attainable. Her beauty was a dead thing;
never by that means could she command homage. But there is love, ay,
{}and passionate love, which can be independent of mere charm of face.
In one man only could she hope to inspire it; successful in that, she
would taste victory, and even in this fallen estate could make for
herself a dominion.

In these few hours she so wrought upon her imagination as to believe
that the one love of her life had declared itself. She revived every
memory she possibly could of those years on the far side of the gulf,
and convinced herself that even then she had loved Sidney. Other love of
a certainty she had not known. In standing face to face with him after
so long an interval, she recognised the qualities which used to impress
her, and appraised them as formerly she could not. His features had
gained in attractiveness; the refinement which made them an index to his
character was more noticeable at the first glance, or perhaps she was
better able to distinguish it. The slight bluntness in his manner
reminded her of the moral force which she had known only as something:
to be resisted: it was now one of the influences that drew her to him.
Had she not always admitted that he stood far above the other men of his
class whom she {}used to know? Between his mind and hers there was
distinct kinship; the sense that he had both power and right to judge
her explained in a great measure her attitude of defiance towards him
when she was determined to break away from her humble conditions. All
along, had not one of her main incentives to work and strive been the
resolve to justify herself in \emph{his} view, to prove to \emph{him}
that she possessed talent, to show herself to \emph{him} as one whom the
world admired? The repugnance with which she thought of meeting him,
when she came home with her father, meant in truth that she dreaded to
be assured that he could only shrink from her.

All her vital force setting in this wild current, her self-deception
complete, she experienced the humility of supreme egoism,---that state
wherein self multiplies its claims to pity in passionate support of its
demand for the object of desire. She felt capable of throwing herself at
Sidney's feet, and imploring him not to withdraw from her the love of
which he had given her so many assurances. She gazed at her scarred face
until the image was blurred with tears; then, as though there were
luxury {}in weeping, sobbed for an hour, crouching down in a corner of
her room. Even though his love were as dead as her beauty, must he not
be struck to the heart with compassion, realising her woeful lot? She
asked nothing more eagerly than to humiliate herself before him, to
confess that her pride was broken. Not a charge he could bring against
her but she would admit its truth. Had she been humble enough last
night? When he came again---and he must soon---she would throw aside
every vestige of dignity, lest he should think that she was strong
enough to bear her misery alone. No matter how poor-spirited she seemed,
if only she could move his sympathies.

Poor rebel heart! Beat for beat, in these moments it matched itself with
that of the purest woman who surrenders to a despairing love. Had one
charged her with insincerity, how vehemently would her conscience have
declared against the outrage! Natures such as hers are as little to be
judged by that which is conventionally the highest standard as by that
which is the lowest. The tendencies which we agree to call good and bad
became {}in her merely directions of a native force which was at all
times in revolt against circumstance. Characters thus moulded may go far
in achievement, but can never pass beyond the bounds of suffering. Never
is the world their friend, nor the world's law. As often as our
conventions give us the opportunity, we crush them out of being; they
are noxious; they threaten the frame of society. Oftenest the crushing
is done in such a way that the hapless creatures seem to have brought
about their own destruction. Let us congratulate ourselves; in one way
or other it is assured that they shall not trouble us long.

Her father was somewhat later than usual in returning from work. When he
entered her room she looked at him anxiously, and as he seemed to have
nothing particular to say, she asked if he had seen Mr. Kirkwood.

``No, my dear, I ain't seen him.''

Their eyes met for an instant. Clara was in anguish at the thought that
another night and day must pass and nothing be altered.

``When did you see him last? A week or more ago, wasn't it?''

``About that.''

{}``Couldn't you go round to his lodgings to-night? I know he's got
something he wants to speak to you about.''

He assented. But on his going into the other room Eagles met him with a
message from Sidney, anticipating his design, and requesting him to step
over to Red Lion Street in the course of the evening. John instantly
announced this to his daughter. She nodded, but said nothing.

In a few minutes John went on his way. The day's work had tired him
exceptionally, doubtless owing to his nervousness, and again on the way
to Sidney's he had recourse to a dose of the familiar stimulant. With
our eyes on a man of Hewett's station we note these little things; we
set them down as a point scored against him; yet if our business were
with a man of leisure, who, owing to worry, found his glass of wine at
luncheon and again at dinner an acceptable support, we certainly should
not think of paying attention to the matter. Poverty makes a crime of
every indulgence. John himself came out of the public-house in a
slinking way, and hoped Kirkwood might not scent the twopenny-worth of
gin.

{}Sidney was in anything but a mood to detect this little lapse in his
visitor. He gave John a chair, but could not sit still himself. The
garret was a spacious one, and whilst talking he moved from wall to
wall.

``You know that I saw Clara last night? She told me she should mention
it to you.''

``Yes, yes. I was afraid she'd never have made up her mind to it. It was
the best way for you to see her alone first, poor girl! You won't mind
comin' to us now, like you used?''

``Did she tell you what she wished to speak to me about?''

``Why, no, she hasn't. Was there---anything particular?''

``She feels the time very heavy on her hands. It seems you don't like
the thought of her looking for employment?''

John rose from his chair and grasped the back of it.

``You ain't a-goin' to encourage her to leave us? It ain't that you was
talkin' about, Sidney?''

``Leave you? Why, where should she go?''

``No, no; it's all right; so long as you wasn't thinkin' of her goin'
away again. See, {}Sidney, I ain't got nothing to say against it, if she
can find some kind of job for home. I know as the time must hang heavy.
There she sit, poor thing! from mornin' to night, an can't get her
thoughts away from herself. It's easy enough to understand, ain't it? I
took a book home for her the other day, but she didn't seem to care
about it. There she sit, with her poor face on her hands, thinkin' and
thinkin'. It breaks my heart to see her. I'd rather she had some work,
but she mustn't go away from home for it.''

Sidney took a few steps in silence.

``You don't misunderstand me,'' resumed the other, with suddenness.
``You don't think as I won't trust her away from me. If she went, it `ud
be because she thinks herself a burden,---as if I wouldn't gladly live
on a crust for my day's food an' spare her goin' among strangers! You
can think yourself what it `ud be to her, Sidney. No, no, it mustn't be
nothino; o' that kind. But I can't bear to see her livin' as she does;
it's no life at all. I sit with her when I get back home at night, an'
I'm glad to say she seems to find it a pleasure to have me by her; but
it's the only bit o' {}pleasure she gets, an' there's all the hours
whilst I'm away. You see she don't take much to Mrs. Eagles; that ain't
her sort of friend. Not as she's got any pride left about her, poor
girl! don't think that. I tell you, Sidney, she's a dear good girl to
her old father. If I could only see her a bit happier, I'd never grumble
again as long as I lived, I wouldn't!''

Is there such a thing in this world as speech that has but one simple
interpretation, one for him who utters it and for him who hears?
Honester words were never spoken than these in which Hewett strove to
represent Clara in a favourable light, and to show the pitifulness of
her situation; yet he himself was conscious that they implied a second
meaning, and Sidney was driven restlessly about the room by his
perception of the same lurking motive in their pathos. John felt
half-ashamed of himself when he ceased; it was a new thing for him to be
practising subtleties with a view to his own ends. But had he said a
word more than the truth?

I suppose it was the association of contrast that turned Sidney's
thoughts to Joseph Snowdon. At all events it was of him he was
{}thinking in the silence that followed. Which silence having been
broken by a tap at the door, oddly enough there stood Joseph himself.
Hewett, taken by surprise, showed embarrassment and awkwardness; it was
always hard for him to reconcile his present subordination to Mr.
Snowdon with the familiar terms on which they had been not long ago.

``Ah, you here, Hewett!'' exclaimed Joseph, in a genial tone, designed
to put the other at his ease. ``I just wanted a word with our friend.
Never mind; some other time.''

For all that, he did not seem disposed to withdraw, but stood with a
hand on the door, smiling. Sidney, having nodded to him, walked the
length of the room, his head bent and his hands behind him.

``Suppose I look in a bit later,'' said Hewett. ``Or to-morrow night,
Sidney?''

``Very well, to-morrow night.''

John took his leave, and on the visitor who remained Sidney turned a
face almost of anger. Mr. Snowdon seated himself, supremely indifferent
to the inconvenience he had probably caused. He seemed in excellent
humour.

``Decent fellow, Hewett,'' he observed, {}putting up one leg against the
fireplace. Very decent fellow. He's getting old, unfortunately. Had a
good deal of trouble, I understand; it breaks a man up.''

Sidney scowled, and said nothing.

``I thought I'd stay, as I \emph{was} here,'' continued Joseph,
unbuttoning his respectable overcoat and throwing it open. ``There was
something rather particular I had in mind. Won't you sit down?''

``No, thank you.''

Joseph glanced at him, and smiled all the more.

``I've had a little talk with the old man about Jane. By the bye, I'm
sorry to say he's very shaky; doesn't look himself at all. I didn't know
you had spoken to him quite so---you know what I mean. It seems to be
his idea that everything's at an end between you.''

``Perhaps so.''

``Well, now, look here. You won't mind me just{{------}}. Do you think
it was wise to put it in that way to him? I'm afraid you're making him
feel just a little uncertain about you. I'm speaking as a friend, you
know. In your own interest, Kirkwood. Old men get queer ideas {}into
their heads. You know, he \emph{might} begin to think that you had some
sort of---eh?''

It was not the second, nor yet the third, time that Joseph had looked in
and begun to speak in this scrappy way, continuing the tone of that
dialogue in which he had assumed a sort of community of interest between
Kirkwood and himself. But the limit of Sidney's endurance was reached.

``There's no knowing,'' he exclaimed, ``what any one may think of me, if
people who have their own ends to serve go spreading calumnies. Let us
understand each other, and have done with it. I told Mr. Snowdon that I
could never be anything but a friend to Jane. I said it, and I meant it.
If you've any doubt remaining, in a few days I hope it'll be removed.
What your real wishes may be I don't know, and I shall never after this
have any need to know. I can't help speaking in this way, and I want to
tell you once for all that there shall never again be a word about Jane
between us. Wait a day or two, and you'll know the reason.''

Joseph affected an air of gravity---of offended dignity.

{}``That's rather a queer sort of way to back out of your engagements,
Kirkwood. I won't say anything about myself, but with regard to my
daughter''{{------}}

``What do you mean by speaking like that?'' cried the young man,
sternly. ``You know very well that it's what you wish most of all, to
put an end to everything between your daughter and me! You've succeeded;
be satisfied. If you've anything to say to me on any other subject, say
it. If not, please let's have done for the present. I don't feel in a
mood for beating about the bush any longer.''

``You've misunderstood me altogether, Kirkwood,'' said Joseph, unable to
conceal a twinkle of satisfaction in his eyes.

``No; I've understood you perfectly well,---too well. I don't want to
hear another word on the subject, and I won't. It's over; understand
that.''

``Well, well; you're a bit out of sorts. I'll say good-bye for the
present.''

He retired, and for a long time Sidney sat in black brooding.

John Hewett did not fail to present {}himself next evening. As he
entered the room he was somewhat surprised at the cheerful aspect with
which Sidney met him; the grasp which his hand received seemed to have a
significance. Sidney, after looking at him steadily, asked if he had not
been home.

``Yes, I've been home. Why do you want to know?''

``Hadn't Clara anything to tell you?''

``No. What is it?''

``Did she know you were coming here?''

``Why, yes; I mentioned it.''

Sidney again regarded him fixedly, with a smile.

``I suppose she preferred that I should tell you. I looked in at the
Buildings this afternoon, and had a talk with Clara.''

John hung upon his words, with lips slightly parted, with a trembling in
the hairs of his grey beard.

``You did?''

``I had something to ask her, so I went when she was likely to be alone.
It's a long while ago since I asked her the question for the first
time,---but I've got the right answer at last.''

{}John stared at him in pathetic agitation.

``You mean to tell me you've asked Clara to marry you?''

``There's nothing very dreadful in that, I should think.''

``Give us your hand again! Sidney Kirkwood, give us your hand again! If
there's a good-hearted man in this world, if there's a faithful, honest
man, as only lives to do kindness{{------}}. What am I to say to you?
It's too much for me. I can't find a word as I'd wish to speak. Stand
out and let's look at you. You make me as I can't neither speak nor
see,---I'm just like a child''{{------}}

He broke down utterly, and shook with the choking struggle of laughter
and sobs. His emotion affected Sidney, who looked pale and troubled in
spite of the smile still clinging feebly about his lips.

``If it makes you glad to hear it,'' said the young man, in an uncertain
voice, ``I'm all the more glad myself, on that account.''

``Makes me glad? That's no word for it, boy; that's no word for it! Give
us your hand again. I feel as if I'd ought to go down on my old knees
and crave your pardon. If {}only she could have lived to see this, the
poor woman as died when things was at their worst! If I'd only listened
to her there'd never have been them years of unfriendliness between us.
You've gone on with one kindness after another, but this is more than I
could ever a' thought possible. Why, I took it for certain as you was
goin' to marry that other young girl; they told me as it was all
settled.''

``A mistake.''

``I'd never have dared to hope it, Sidney. The one thing as I wished
more than anything else on earth, and I couldn't think ever to see it.
Glad's no word for what I feel. And to think as my girl kep' it from me!
Yes, yes; there was something on her face; I remember it now. `I'm just
goin' round to have a word with Sidney,' I says. `Are you, father?' she
says. `Don't stay too long.' And she had a sort o' smile I couldn't
quite understand. She'll be a good wife to you, Sidney. Her heart's
softened to all as she used to care for. She'll be a good and faithful
wife to you as long as she lives. But I must go back home and speak to
her. There ain't a man {}livin', let him be as rich as he may, that
feels such happiness as you've given me to-night.''

He went stumbling down the stairs, and walked homewards at a great
speed, so that when he reached the Buildings he had to wipe his face and
stand for a moment before beginning the ascent. The children were at
their home lessons; he astonished them by flinging his hat mirthfully on
to the table.

``Now then, father!'' cried young Tom, the eight-year-old, whose pen was
knocked out of his hand. With a chuckle John advanced to Clara's room.
As he closed the door behind him she rose. His face was mottled; there
were tear-stains about his eyes, and he had a wild, breathless look.

``An' you never told me! You let me go without half a word!''

Clara put her hands upon his shoulders and kissed him. ``I didn't quite
know whether it was true or not, father.''

``My darling! My dear girl! Come an' sit on my knee, like you used to
when you was a little 'un. I'm a rough old father for such as you, but
nobody'll never love you {}better than I do, an always have done. So
he's been faithful to yon, for all they said. There ain't a better man
livin'! `It's a long time since I first asked the question,' he says,
but she's give me the right answer at last.' And he looks that glad of
it.''

``He does? You're sure he does?''

``Sure? Why, you should a' seen him when I went into the room! There's
nothing more as I wish for now. I only hope I may live a while longer,
to see you forget all your troubles, my dear. He'll make you happy, will
Sidney; he's got a deal more education than any one else I ever knew,
and you'll suit each other. But you won't forget all about your old
father? You'll let me come an' have a talk with you now and then, my
dear, just you an' me together, you know?''

``I shall love you and be grateful to you always, father. You've kept a
warm heart for me all this time.''

``I couldn't do nothing else, Clara; you've always been what I loved
most, and you always will be.''

``If I hadn't had you to come back to, what would have become of me?''

{}``We'll never think of that. We'll never speak another word of that.''

``Father{{------}} Oh, if I had my face again! If I had my own face!''

A great anguish shook her; she lay in his arms and sobbed. It was the
farewell, even in her fulness of heart and deep sense of consolation, to
all she had most vehemently desired. Gratitude and self-pity being
indivisible in her emotions, she knew not herself whether the ache of
regret or the soothing: restfulness of deliverance made her tears flow.
But at least there was no conscious duplicity, and for the moment no
doubt that she had found her haven. It is a virtuous world, and our
frequent condemnations are invariably based on justice; will it be
greatly harmful if for once we temper our righteous judgment with ever
so little mercy?

\chapter{\emph{``Io Saturnalia!''}}

\textsc{So} at length came Monday, the first Monday in August, a day
gravely set apart for the repose and recreation of multitudes who
neither know how to rest nor how to refresh themselves with pastime.
To-day will the slaves of industrialism don the \emph{pileus}. It is
high summertide. With joy does the awaking publican look forth upon the
blue-misty heavens, and address his adorations to the Sun-god, inspirer
of thirst. Throw wide the doors of the temple of Alcohol! Behold, we
come in our thousands, jingling the coins that shall purchase us this
one day of tragical mirth. Before us is the dark and dreary autumn; it
is a far cry to the foggy joys of Christmas, Io Saturnalia!

For certain friends of ours this morning {}brought an event of
importance. At a church in Clerkenwell were joined together in holy
matrimony Robert Hewett and Penelope (otherwise Pennyloaf) Candy, the
former aged nineteen, the latter less than that by nearly three years.
John Hewett would have nothing to do with an alliance so disreputable;
Mrs. Hewett had in vain besought her stepson not to marry so unworthily.
Even as a young man of good birth has been known to enjoy a subtle
self-flattery in the thought that he graciously bestows his name upon a
maiden who, to all intents and purposes, may be said never to have been
born at all, so did Bob Hewett feel when he put a ring upon the scrubby
finger of Pennyloaf. Proudly conscious was Bob that he had ``married
beneath him,''---conscious also that Clem Peckover was gnawing her lips
in rage.

Mrs. Candy was still sober at the hour of the ceremony. Her husband, not
a bad fellow in his way, had long since returned to her, and as yet had
not done more than threaten a repetition of his assault. Both were
{}sent at church. A week ago Bob had established himself in a room in
Shooter's Gardens, henceforth to be shared with him by his bride.
Probably he might have discovered a more inviting abode for the early
days of married life, but Bob had something of the artist's temperament
and could not trouble about practical details; for the present this room
would do as well as another. It was cheap, and he had need of all the
money he could save from everyday expenses. Pennyloaf would go on with
her shirtmaking, of course, and all they wanted was a roof over their
heads at night.

And in truth he was fond of Pennyloaf. The poor little slave worshipped
him so sincerely; she repaid his affectionate words with such fervent
gratitude; and there was no denying that she had rather a pretty face,
which had attracted him from the first. But above all, this preference
accorded to so humble a rival had set Clem Peckover beside herself. It
was all very well for Clem to make pretence of having transferred her
{}affections to Jack Bartley. Why, Suke Jollop (ostensibly Clem's bosom
friend, but treacherous at times because she had herself given an eye to
Jack)---Suke Jollop reported that Clem would have killed Pennyloaf had
she dared. Pennyloaf had been going about in fear for her life since
that attack upon her in Myddelton Passage. ``I dursn't marry you, Bob! I
dursn't!'' she kept saying, when the proposal was first made. But Bob
laughed with contemptuous defiance. He carried his point, and now he was
going to spend his wedding-day at the Crystal Palace,---choosing that
resort because he knew Clem would be there, and Jack Bartley, and Suke
Jollop, and many another acquaintance, before whom he was resolved to
make display of magnanimity. Pennyloaf shone in most unwonted apparel.
Everything was new except her boots,---it had been decided that these
only needed soleing. Her broad-brimmed hat of yellow straw was graced
with the reddest feather purchasable in the City Road; she had a dolman
of most fashionable cut, blue, lustrous; blue {}likewise was her dress,
hung about with bows and streamers. And the gleaming ring on the scrubby
small finger! On that hand most assuredly Pennyloaf would wear no glove.
How proud she was of her ring! How she turned it round and round when
nobody was looking! Gold, Pennyloaf, real gold! The pawnbroker would
lend her seven-and-sixpence on it, any time. At Holborn Viaduct there
was a perpetual rush of people for the trains to the ``Paliss.''

As soon as a train was full, off it went, and another long string of
empty carriages drew up in its place. No distinction between ``classes''
to-day; get in where you like, where you can. Positively, Pennyloaf
found herself seated in a first-class carriage; she would have been
awe-struck, but that Bob flung himself back on the cushions with such an
easy air and nodded laughingly at her. Among their companions was a
youth with a concertina; as soon as the train moved he burst into
melody. It was the natural invitation to song, and all joined in the
latest {}ditties learnt at the music-hall. Away they sped, over the
roofs of south London, about them the universal glare of sunlight, the
carriage dense with tobacco smoke. Ho for the bottle of muddy ale,
passed round in genial fellowship from mouth to mouth! Pennyloaf would
not drink of it; she had a dread of all such bottles. In her heart she
rejoiced that Bob knew no craving for strong liquor. Towards the end of
the journey the young man with the concertina passed round his hat.

Clem Peckover had come by the same train; she was one of a large party
which had followed close behind Bob and Pennyloaf to the railway
station. Now they followed along the long corridors into the ``Paliss,''
with many a loud expression of mockeiy, with hee-hawing laughter, with
coarse jokes. Depend upon it, Clem was gorgeously arrayed; amid her
satellites she swept on ``like a stately ship of Tarsus, bound for the
isles of Javan or Gadire;'' her face was aflame, her eyes flashed in
enjoyment of the {}uproar. Jack Bartley wore a high hat,---Bob never had
owned one in his life,---and about his neck was a tie of crimson; yellow
was his waistcoat, even such a waistcoat as you may see in Pall Mall,
and his walking-stick had a nigger's head for handle. He was the oracle
of the maidens around him; every moment the appeal was to ``Jeck!
Jeck!'' Suke Jollop, who would in reality have preferred to accompany
Bob and his allies, whispered it about that Jack had two-pound-ten in
his pocket, and was going to spend every penny of it before he left the
``Paliss,''---yes, ``every bloomin' penny!''

Thus early in the day, the grounds were of course preferred to the
interior of the glass house. Bob and Pennyloaf bent their steps to the
fair. Here abeady was gathered much goodly company; above their heads
hung a thick white wavering cloud of dust. Swingboats and
merry-go-rounds are from of old the chief features of these rural
festivities; they soared and dipped and circled to the joyous music of
organs which played the {}same tune automatically for any number of
hours, whilst raucous voices invited all and sundry to take their turn.
Should this delight pall, behold on every hand such sports as are
dearest to the Briton, those which call for strength of sinew and
exactitude of aim. The philosophic mind would have noted with interest
how ingeniously these games were made to appeal to the patriotism of the
throng. Did you choose to ``shy'' sticks in the contest for cocoa-nuts,
behold your object was a wooden model of the treacherous Afghan or the
base African. If you took up the mallet to smite upon a spring and make
proof of how far you could send a ball flying upwards, your blow
descended upon the head of some other recent foeman. Try your fist at
the indicator of muscularity, and with zeal you smote full in the
stomach of a guy made to represent a Russian. If you essayed the
pop-gun, the mark set you was on the flank of a wooden donkey, so
contrived that it would kick when hit in the true spot. What a joy to
observe the {}tendency of all these diversions! How characteristic of a
high-spirited people that nowhere could be found any amusement appealing
to the mere mind, or calculated to effeminate by encouraging a love of
beauty.

Bob had a sovereign to get rid of. He shied for cocoa-nuts, he swung in
the boat with Pennyloaf, he rode with her on the whirligigs. When they
were choked, and whitened from head to foot, with dust, it was natural
to seek the nearest refreshment-booth. Bob had some half-dozen male and
female acquaintances clustered about him by now; of course he must
celebrate the occasion by entertaining all of them. Consumed with
thirst, he began to drink without counting the glasses. Pennyloaf
plucked at his elbow, but Bob was beginning to feel that he must display
spirit. Because he was married, that was no reason for his relinquishing
the claims to leadership in gallantry which had always been recognised.
Hollo! Here was Suke Jollop! She had just quarrelled with Clem, and had
been searching for the hostile camp. {}``Have a drink, Suke!'' cried
Bob, when he heard her acrimonious charges against Clem and Jack. A
pretty girl, Suke, and with a hat which made itself proudly manifest a
quarter of a mile away. Drink! of course she would drink; that thirsty
she could almost drop! Bob enjoyed this secession from the enemy. He
knew Suke's old fondness for him, and began to play upon it. Elated with
beer and vanity, he no longer paid the least attention to Pennyloaf's
remonstrances; nay, he at length bade her ``hold her bloomin' row!''
Pennyloaf had a tear in her eye; she looked fiercely at Miss Jollop.

The day wore on. For utter weariness Pennyloaf was constrained to beg
that they might go into the ``Paliss'' and find a shadowed seat. Her
tone revived tenderness in Bob; again he became gracious, devoted; he
promised that not another glass of beer should pass his lips, and Suke
Jollop, with all her like, might go to perdition. But heavens! how
sweltering it was under this glass canopy! How the dust rose from the
trampled boards! {}Come, let's have tea. The programme says there'll be
a militaiy band playing presently, and we shall return refreshed to hear
it.

So they made their way to the ``Shilling Tea-room.'' Having paid at the
entrance, they were admitted to feed freely on all that lay before them.
With difficulty could a seat be found in the huge room; the uproar of
voices was deafening. On the tables lay bread, butter, cake in
huncheons, tea-pots, milk-jugs, sugar-basins,---all things to whomso
could secure them in the conflict. Along the gangways coursed perspiring
waiters, heaping up giant structures of used plates and cups,
distributing clean utensils, and miraculously sharp in securing the
gratuity expected from each guest as he rose satiate. Muscular men in
aprons wheeled hither the supplies of steaming fluid in immense cans on
heavy trucks. Here practical joking found the most graceful of
opportunities, whether it were the deft direction of a piece of cake at
the nose of a person sitting opposite, or the emptying of a saucer down
your neighbour's back, or the ingenious {}jogging of an arm which was in
the act of raising a full tea-cup. Now and then an ill-conditioned
fellow, whose beer disagreed with him, would resent some piece of
elegant trifling, and the waiters would find it needful to request
gentlemen not to fight until they had left the room. These cases,
however, were exceptional. On the whole there reigned a spirit of
imbecile joviality. Shrieks of female laughter testified to the success
of the entertainment

As Bob and his companion quitted this sphere of delight, ill-luck
brought it to pass that Mr. Jack Bartley and his train were on the point
of entering. Jack uttered a phrase of stinging sarcasm with reference to
Pennyloaf's red feather; whereupon Bob smote him exactly between the
eyes. Yells arose; there was a scuffle, a rush, a tumult. The two were
separated before further harm came of the little misunderstanding, but
Jack went to the tea-tables vowing vengeance.

Poor Pennyloaf shed tears as Bob led her to the place where the band had
begun {}playing. Only her husband's anger prevented her from yielding to
utter misery. But now they had come to the centre of the building, and
by dint of much struggle in the crowd they obtained a standing whence
they could see the vast amphitheatre, filled with thousands of faces.
Here at length was quietness, intermission of folly and brutality. Bob
became another man as he stood and listened. He looked with kindness
into Pennyloafs pale, weary face, and his arm stole about her waist to
support her. Ha! Pennyloaf was happy! The last trace of tears vanished.
She too was sensible of the influences of music; her heart throbbed as
she let herself lean against her husband.

Well, as every one must needs have his panacea for the ills of society,
let me inform you of mine. To humanise the multitude two things are
necessary,---two things of the simplest kind conceivable. In the first
place, you must effect an entire change of economic conditions: a
preliminary step of which every tiro will recognise the easiness; then
you {}must bring to bear on the new order of things the constant
influence of music. Does not the prescription recommend itself? It is
jesting in earnest. For, work as you will, there is no chance of a new
and better world until the old be utterly destroyed. Destroy, sweep
away, prepare the ground; then shall music the holy, music the
civiliser, breathe over the renewed earth and with Orphean magic raise
in perfected beauty the towers of the City of Man.

Hours yet before the fireworks begin. Never mind; here by good luck we
find seats where we can watch the throng passing and repassing. It is a
great review of the People. On the whole how respectable they are, how
sober, how deadly dull! See how worn-out the poor girls are becoming,
how they gape, what listless eyes most of them have! The stoop in the
shoulders so universal among them merely means over-toil in the
workroom. Not one in a thousand shows the elements of taste in dress;
vulgarity and worse glares in all but every costume. Observe the
{}middle-aged women; it would be small surprise that their good-looks
had vanished, but whence comes it they are animal, repulsive, absolutely
vicious in ugliness? Mark the men in their turn; four in every six have
visages so deformed by ill-health that they excite disgust; their hair
is cut down to within half an inch of the scalp; their legs are twisted
out of shape by evil conditions of life from birth upwards. Whenever a
youth and a girl come along arm-in-arm, how flagrantly shows the man's
coarseness! They are pretty, so many of these girls, delicate of
feature, graceful did but their slavery allow them natural development;
and the heart sinks as one sees them side by side with the men who are
to be their husbands.

One of the livelier groups is surging hitherwards; here we have frolic,
here we have humour. The young man who leads them has been going about
all day with the lining of his hat turned down over his forehead; for
the thousandth time those girls are screaming with laughter at the sight
of him. Ha ha! {}He has slipped and fallen upon the floor, and makes an
obstruction; his companions treat him like a horse that is ``down'' in
the street.

``Look out for his `eels!'' cries one; and another, ``Sit on his 'ed!''
If this doesn't come to an end we shall die of laughter. Lo! one of the
funniest of the party is wearing a gigantic cardboard nose and
flame-coloured whiskers. There, the stumbler is on his feet again.
``\,`Ere he comes up smilin'!'' cries his friend of the cardboard nose,
and we shake our diaphragms with mirth. One of the party is an unusually
tall man. ``When are you comin' down to have a look at us?'' cries a
pert lass as she skips by him.

A great review of the People. Since man came into being, did the world
ever exhibit a sadder spectacle?

Evening advances; the great ugly building will presently be lighted with
innumerable lamps. Away to the west yonder the heavens are afire with
sunset, but at that we do not care to look; never in our lives did we
regard it. We know not what is meant by beauty {}or grandeur. Here under
the glass roof stand white forms of undraped men and women,---casts of
antique statues,---but we care as little for the glory of art as for
that of nature; we have a vague feeling that, for some reason or other,
antiquity excuses the indecent, but further than that we do not get.

As the dusk descends there is a general setting of the throng towards
the open air; all the pathways swarm with groups which have a tendency
to disintegrate into couples; universal is the protecting arm. Relief
from the sweltering atmosphere of the hours of sunshine causes a revival
of hilarity; those who have hitherto only bemused themselves with liquor
now pass into the stage of jovial recklessness, and others, determined
to prolong a flagging merriment, begin to depend upon their companions
for guidance. On the terraces dancing has commenced; the players of
violins, concertinas, and penny-whistles do a brisk trade among the
groups eager for a rough-and-tumble valse; so do the {}pick-pockets.
Vigorous and varied is the jollity that occupies the external galleries,
filling now in expectation of the fireworks; indescribable the mingled
tumult that roars heavenwards. Girls linked by the half-dozen arm-inarm
leap along with shrieks like grotesque mcenads; a rougher horseplay
finds favour among the youths, occasionally leading to fisticuffs. Thick
voices bellow in fragmentary chorus; from every side comes the yell, the
cat-call, the ear-rending whistle; and as the bass, the never-ceasing
accompaniment, sounds myriad-footed tramp, tramp along the wooden
flooring. A fight, a scene of bestial drunkenness, a tender whispering
between two lovers, proceed concurrently in a space of five square
yards.---Above them glimmers the dawn of starlight.

For perhaps the first time in his life Bob Hewett has drunk more than he
can well carry. To Pennyloaf's remonstrances he answers more and more
impatiently: ``Why does she talk like a bloomin' fool?---one doesn't get
married every day.'' He is on the {}lookout for Jack Bartley now; only
let him meet Jack, and it shall be seen who is the better man. Pennyloaf
rejoices that the hostile party are nowhere discoverable. She is
persuaded to join in a dance, though every moment it seems to her that
she must sink to the ground in uttermost exhaustion. Naturally she does
not dance with sufficient liveliness to please Bob; he seizes another
girl, a stranger, and whirls round the six-foot circle with a laugh of
triumph. Pennyloaf's misery is relieved by the beginning of the
fireworks. Up shoot the rockets, and all the reeking multitude utters a
huge ``Oh!'' of idiot admiration.

Now at length must we think of tearing ourselves away from these
delights. Akeady the more prudent people are hurrying to the railway,
knowing by dire experience what it means to linger until the last
cargoes. Pennyloaf has hard work to get her husband as far as the
station; Bob is not quite steady upon his feet, and the hustling of the
crowd perpetually excites him to bellicose challenges. {}They reach the
platform somehow; they stand wedged amid a throng which roars
persistently as a substitute for the activity of limb Row become
impossible. A train is drawing up slowly; the danger is lest people in
the front row should be pushed over the edge of the platform, but
porters exert themselves with success. A rush, a tumble, curses, blows,
laughter, screams of pain,---and we are in a carriage. Pennyloaf has to
be dragged up from under the seat, and all her indignation cannot free
her from the jovial embrace of a man who insists that there is plenty of
room on his knee. Off we go! It is a long third-class coach, and already
five or six musical instruments have struck up. We smoke and sing at the
same time; we quarrel and make love,---the latter in somewhat primitive
fashion; we roll about with the rolling of the train; we nod into
hoggish sleep.

The platform at Holborn Viaduct; and there, to Pennyloaf's terror, it is
seen that Clem Peckover and her satellites have come {}by the same
train. She does her best to get Bob quickly away, but Clem keeps close
in their neighbourhood. Just as they issue from the station, Pennyloaf
feels herself bespattered from head to foot with some kind of fluid;
turning, she is aware that all her enemies have squirts in their hands,
and are preparing for a second discharge of filthy water. Anguish for
the ruin of her dress overcomes all other fear; she calls upon Bob to
defend her.

But an immediate conflict was not Jack Bartley's intention. He and those
with him made off at a run, Bob pursuing as closely as his unsteadiness
would permit. In this way they all traversed the short distance to
Clerkenwell Green, either party echoing the other's objurgations along
the thinly-peopled streets. At length arrived the suitable moment. Near
St. James's Church Jack Bartley made a stand, and defied his enemy to
come on. Bob responded with furious eagerness; amid a press of delighted
spectators, swelled by people just turned out of the public-houses, the
two lads {}fought like wild animals. Nor were they the only combatants.
Exasperated by the certainty that her hat and dolman were ruined,
Pennyloaf flew with erected nails at Clem Peckover. It was just what the
latter desired; in an instant she had rent half Pennyloaf's garments off
her back, and was tearing her face till the blood streamed. Inconsolable
was the grief of the crowd when a couple of stalwart policemen came
hustling forward, thrusting to left and right, irresistibly clearing the
corner. There was no question of making arrests; it was the night of
Bank-holiday, and the capacity of police-cells is limited. Enough that
the fight perforce came to an end. Amid frenzied blasphemy Bob and Jack
went their several ways; so did Clem and Pennyloaf.

Poor Pennyloaf! Arrived at Shooter's Gardens, and having groped her way
blindly up to the black hole which was her wedding-chamber, she just
managed to light a candle, then sank down upon the bare floor and wept.
You could not have recognised her; her pretty face was all blood and
dirt. She held in her {}hand the fragment of a hat, and her dolman had
disappeared. Her husband was not in much better plight; his waistcoat
and shirt were rent open, his coat was filth-smeared, and it seemed
likely that he had lost the sight of one eye. Sitting there in drunken
lassitude, he breathed nothing but threats of future vengeance.

An hour later noises of a familiar kind sounded beneath the window. A
woman's voice was raised in the fury of mad drunkenness, and a man
answered her with threats and blows.

``That's mother,'' sobbed Pennyloaf. ``I knew she wouldn't get over
to-day. She never did get over a Bank-holiday.''

Mrs. Candy had taken the pledge when her husband consented to return and
live with her. Unfortunately she did not at the same time transfer
herself to a country where there are no beer-shops and no Bank-holidays.
Short of such decisive change, what hope for her?

Bob was already asleep, breathing {}stertorously. As for Pennyloaf, she
was so over-wearied that hours passed before oblivion fell upon her
aching eyelids. She was thinking all the time that on the morrow it
would be necessary to pawn her wedding-ring.

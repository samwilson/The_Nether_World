\chapter{Clara and Jane}

\textsc{Rain} no longer fell, but the gusty and bitter wind still swept
about the black streets. Walking side by side without speech, Clara and
her companion left the neighbourhood of the prison and kept a northward
direction till they reached the junction of highways where stands the
``Angel.'' Here was the wonted crowd of loiterers and the press of
people waiting for tramcar or omnibus, east, west, south, or north;
newsboys, eager to get rid of their last batch, were crying as usual,
``Ech-ow! Exteree speciul! Ech-ow! Steendard!'' and a brass band was
blaring out its saddest strain of merry dance-music. The lights gleamed
dismally in rain-puddles and on the wet pavement. With the wind {}came
whiffs of tobacco and odours of the drinking-bar.

They crossed, and walked the length of Islington High Street, then a
short way along its continuation, Upper Street. Once or twice Clara had
barely glanced at Kirkwood, but his eyes made no reply, and his lips
were resolutely closed. She did not seem offended by this silence; on
the contrary, her face was cheerful and she smiled to herself now and
then. One would have imagined that she found pleasure in the sombreness
of which she was the cause.

She stopped at length, and said:

``I suppose you don't want to go in with me?''

``No.''

``Then I'll say good-night. Thank you for coming so far out of your
way.''

``I'll wait. I may as well walk back with you, if you don't mind.''

``Oh, very well. I shan't be many minutes.''

She passed on and entered the place of {}refreshment that was kept by
Mrs. Tubbs. Till recently it had been an ordinary eatinghouse or
coffee-shop; but, having succeeded in obtaining a license to sell strong
liquors, Mrs. Tubbs had converted the establishment into one of a more
pretentious kind. She called it ``Imperial Restaurant and Luncheon
Bar.'' The front shone with vermilion paint; the interior was aflare
with many gas-jets; in the window was disposed a tempting exhibition of
``snacks'' of fish, cold roast fowls, ham-sandwiches, and the like;
whilst farther back stood a cooking-stove whereon frizzled and vapoured
a savoury mess of sausages and onions.

Sidney turned away a few paces. The inclemency of the night made Upper
Street---the promenade of a great district on account of its spacious
pavement---less frequented than usual, but there were still numbers of
people about, some hastening homewards, some sauntering hither and
thither in the familiar way, some gathered into gossipping groups.
Kirkwood was irritated by the {}conversation and laughter that fell on
his ears, irritated by the distant strains of the band, irritated above
all by the fume of frying that pervaded the air for many yards about
Mrs. Tubbs's precincts. He observed that the customers tending that way
were numerous; they consisted mainly of lads and young men who had come
forth from neighbouring places of entertainment. The locality and its
characteristics had been familiar to him from youth upwards, but his
nature was not subdued to what it worked in, and the present fit of
disgust was only an accentuation of a mood by which he was often
possessed. To the Hewetts he had spoken impartially of Mrs. Tubbs and
her bar; probably that was the right view; but now there came back upon
him the repugnance with which he had regarded Clara's proposal when it
was first made.

It seemed to him that he had waited nearly half an hour when Clara came
forth again. In silence she walked on beside him. Again they crossed by
the ``Angel'' and entered St. John Street Road.

{}``You've made your arrangements?'' Sidney said, now that there were
few people passing.

``Yes; I shall go on Monday.''

``You're going to live there altogether?''

``Yes; it'll be more convenient, and then it'll give them more room at
home. Bob can sleep with the children, and save money.''

``To be sure!'' observed the young man with bitter irony. Clara flashed
a glance at him. It was a new thing for Sidney to take this tone with
her; not seldom he had expressed unfavourable judgments by silence, but
he had never spoken to her otherwise than with deference and gentleness.

``You don't seem in a very good temper to-night, Mr. Kirkwood,'' she
remarked in a suave tone. He disregarded her words, but in a few moments
turned upon her and said scornfully:

"I hope you'll enjoy the pleasant ladylike work you've found! I should
think it'll {}improve your self-respect to wait on the gentlemen of
Upper Street!"

Irony is not a weapon much in use among working-people; their wits in
general are too slow. With Sidney, however, it had always been a habit
of speech in indignant criticism, and sympathy made him aware that
nothing would sting Clara more acutely. He saw that he was successful
when she turned her head away and moved it nervously.

``And do you suppose I go there because the place pleases me?'' she
asked in a cold, hostile voice. "You make a great mistake, as you always
do when you pretend to know anything about \emph{me}. Wait till I've
learned a little about the business; you won't find me in Upper Street
then."

``I understand.''

Again they walked on in silence. They were nearing Clerkenwell Close,
and had to pass a corner of the prison in a dark lane, where the wind
moaned drearily. The line of the high blank wall was relieved in
colourless gloom against a sky of sheer night. {}Opposite, the shapes of
poverty-eaten houses and grimy workshops stood huddling in the
obscurity. From near at hand came shrill voices of children chasing each
other about---children playing at midnight between slum and gaol!

``We're not likely to see much of each other after to-night,'' said
Sidney, stopping.

``The less the better, I should say, if this is how you're going to talk
to me.''

"The less the better, perhaps,---at all events for a time. But there's
one or two things on my mind, and I'll say them now. I don't know
whether you think anything about it, but you must have seen that things
are getting worse and worse at home. Your mother{{------}}"

``She's no mother of mine! '' broke in Clara angrily.

"She's been a mother to you in kindness, that's certain, and you've
repaid her almost as ill as you could have done. Another girl would have
made her hard life a bit easier. No; you've only thought of yourself.
Your {}father walks about day after day trying to get work, and how do
you meet him when he comes home? You fret him and anger him; you throw
him back ill-tempered words when he happens to think different from you;
you almost break his heart, because you won't give way in things that he
only means for your good---he that would give his life for you! It's as
well you should hear the truth for once, and hear it from me, too. Any
one else might speak from all sorts of motives; as for me, it makes me
suffer more to say such things than it ever could you to hear them.
Laugh if you like! I don't ask you to pay any heed to what \emph{I}'ve
wished and hoped; but just give a thought to your father, and the rest
of them at home. I told him to-night he'd only to trust you, that you
never could do anything to make him ashamed of you. I said so, and I
believe it. Look, Clara! with all my heart I believe it. But now you've
got your way, think of them a little."

``It isn't your fault if I don't know how bad I am,'' said the girl with
a half smile. {}That she did not resent his lecture more decidedly was
no doubt due to its having afforded new proof of the power she had over
him. Sidney was shaken with emotion; his voice all but failed him at the
last.

``Good-bye,'' he said, turning away.

Clara hesitated, looked at him, but finally also said ``Good-bye,'' and
went on alone.

She walked with bent head, and almost passed the house-door in absence
of thought. On the threshold was standing Miss Peckover; she drew aside
to let Clara pass. Between these two was a singular rivalry. Though by
date a year younger than Clara, Clem gave no evidence of being
physically less mature. In the matter of personal charms she regarded
herself as by far Miss Hewett's superior, and resented vigorously the
tone of the latter's behaviour to her. Clara, on the other hand, looked
down upon Miss Peckover as a mere vulgar girl; she despised her brother
Bob because he had allowed himself to be inveigled by Clem; in
intellect, in social standing, she considered {}herself out of all
comparison with the landlady's daughter. Clem had the obvious advantage
of being able to ridicule the Hewetts' poverty, and did so without
sparing. Now, for instance, when Clara was about to pass with a distant
``Good-night,'' Clem remarked:

``It's cold, ain't it? I wonder you don't put on a ulster, a night like
this.''

``Thank you,'' was the reply. ``I shan't consult you about how I'm to
dress.''

Clem laughed, knowing she had the best of the joke.

The other went upstairs, and entered the back-room, where it was quite
dark.

``That you, Clara?'' asked Amy's voice. ``The candle's on the
mantel-shelf''

``Why aren't you asleep?'' Clara returned sharply. But the irritation
induced by Clem's triumph quickly passed in reflection on Sidney's mode
of leave-taking. That had not at all annoyed her, but it had made her
thoughtful. She lit the candle. Its light disclosed a room much barer
than the other one. There was one bed, in which Amy and Annie {}lay
(Clara had to share it with them), and a mattress placed on the floor,
where reposed little Tom; a low chest of drawers with a very small
looking-glass upon it, a washstand, a few boxes. Handsome girls,
unfortunate enough to have brains to boot, do not cultivate the patient
virtues in chambers of this description.

There was a knock at the door. Clara found her father standing there.

``Have you anything to tell me, my girl?'' he asked in a subdued voice,
furtively regarding her.

``I shall go on Monday.''

He drew back a step, and seemed about to return to the other room.

``Father, I shall have to give Mrs. Tubbs the five shillings for a few
weeks. She's going to let me have a new dress.''

``Your earnin's is your own, Clara.''

``Yes, but I hope very soon to be able to give you something. Its hard
for you, having no work.''

John brightened wonderfully.

{}``Don't you trouble, my dear. That's all right. Things'll come round
somehow. You're a good girl. Good-night, my darlin!''

He kissed her, and went consoled to his rest.

~

Miss Peckover kept going up and down between the kitchen and the
front-door. Down below, Jane was cleaning a copper kettle. Clem, who had
her sweetest morsel of cruelty yet in store, had devised this pleasant
little job as a way of keeping the child employed till all was quiet.
She had just come down to watch the progress of the work, and to give a
smart rap or two on the toiling fingers, when a heavy footstep in the
passage caused her to dart upstairs again. It was Bob Hewett, returned
from his evening recreations.

``Oh, that's you, is it?'' cried Clem. ``Come down; I want to speak to
you.''

``Wait till to-morrow,'' answered Bob, advancing towards the stairs.

``Wait! we'll see about that!''

She sprang forward, and with a prompt {}exertion of muscle, admirable in
its way, whirled Bob round and dragged him to the head of the kitchen
flight. The young fellow took it in good part, and went down with her.

``You go up into the passage,'' said Clem to her servant, and was
immediately obeyed.

``Now,'' resumed Miss Peckover, when she had closed the door, ``who have
you been goin' about with to-night?''

``What are you talking about?'' returned Bob, who had seated himself on
the table, and was regarding Clem jocosely. ``I've been with some pals,
that's all.''

``Pals! what sort o' pals? Do you call Pennyloaf Candy one o' your
pals?''

She stood before him in a superb attitude, her head poised fiercely, her
arms quivering at her sides, all the stature and vigour of her young
body emphasised by muscular strain.

``Pennyloaf Candy!'' Bob repeated, as if in scorn of the person so
named. ``Get on with you! I'm sick of hearing you talk about her. Why, I
haven't seen her not these three weeks.''

{}"It's a {{------}} lie!" Clem's epithet was too vigorous for
reproduction. ``Sukey Jollop saw you with her down by the meat-market,
an' Jeck Bartley saw you too.''

``Jeck did?'' He laughed with obstreperous scorn. ``Why, Jeck's gone to
Homerton to his mother till Saturday night. Don't be such a bloomin'
fool! Just because Suke Jollop's dead nuts on me, an' I won't have
nothin' to say to her, she goes tellin' these bloomin' lies. When I see
her next, I'll make her go down on her marrow-bones an' beg my pardon.
See if I don't just!''

There was an engaging frankness in Bob's way of defending himself which
evidently impressed Miss Peckover, though it did not immediately soothe
her irritation. She put her arms akimbo, and examined him with a steady
suspicion which would have disconcerted most young men. Bob, however,
only laughed more heartily. The scene was prolonged. Bob had no recourse
to tenderness to dismiss the girl's jealousy. His self-conceit was
supreme, and had always stood him {}in such stead with the young ladies
who, to use his own expression, were ``dead nuts on him,'' that his
love-making, under whatever circumstances, always took the form of
genial banter \emph{de haul en bas}. ``Don't be a bloomin' fool!'' was
the phrase he deemed of most efficacy in softening the female heart; and
the result seemed to justify him, for after some half-hour's wrangling,
Clem abandoned her hostile attitude, and eyed him with a savage kind of
admiration.

``When are you goin' to buy me that locket, Bob, to put a bit of your
'air in?'' she inquired pertinently.

``You just wait, can't you? There's a event coming off next week. I
won't say nothing, but you just wait.''

``I'm tired o' waitin . See `ere; you ain't goin' to best me out of
it?''

``Me best you? Don't be a bloomin' fool, Clem!''

He laughed heartily, and in a few minutes allowed himself to be embraced
and sent off to his chamber at the top of the house.

{}Clem summoned her servant from the passage. At the same moment there
entered another lodger, the only one whose arrival Clem still awaited.
His mode of ascending the stairs was singular; one would have imagined
that he bore some heavy weight, for he proceeded very slowly, with a
great clumping noise, surmounting one step at a time in the manner of a
child. It was Mr. Marple, the cab-driver, and his way of going up to bed
was very simply explained by the fact that a daily sixteen hours of
sitting on the box left his legs in a numb and practically useless
condition.

The house was now quiet. Clem locked the front-door and returned to the
kitchen, eager with anticipation of the jest she was going to carry out.
First of all she had to pick a quarrel with Jane; this was very easily
managed. She pretended to look about the room for a minute, then asked
fiercely:

``What's gone with that sixpence I left on the dresser?''

Jane looked up in terror. She was worn {}almost to the last point of
endurance by her day and night of labour and agitation. Her face was
bloodless, her eyelids were swollen with the need of sleep.

``Sixpence!'' she faltered, ``I'm sure I haven't seen no sixpence,
miss.''

"You haven't? Now, I've caught you at last. There's been nobody `ere but
you. Little thief! We'll see about this in the mornin', an' to-night
\emph{you shall sleep in the back-kitchen!}"

The child gasped for breath. The terror of sudden death could not have
exceeded that which rushed upon her heart when she was told that she
must pass her night in the room where lay the coffin.

``An' you shan't have no candle, neither,'' proceeded Clem, delighted
with the effect she was producing. ``Come along! I'm off to bed, an'
I'll see you safe locked in first, so as no one can come an' hurt you.''

``Miss! please!---I can't, I durstn't!''

Jane pleaded in inarticulate anguish. But Clem had caught her by the
arm, was {}dragging her on, on, till she was at the very door of that
ghastly death-cellar. Though thirteen years old, her slight frame was as
incapable of resisting Clem Peckover's muscles as an infant's would have
been. The door was open, but at that moment Jane uttered a shriek which
rang and echoed through the whole house. Startled, Clem relaxed her
grasp. Jane tore herself away, fled up the kitchen-stairs, fled upwards
still, flung herself at the feet of some one who had come out onto the
landing and held a light.

``Oh, help me! Don't let her! Help me!''

``What's up with you, Jane?'' asked Clara, for it was she who, not being
yet in bed, had come forth at once on hearing the scream.

Jane could only cling to her garment, pant hysterically, repeat the same
words of entreaty again and again. Another door opened, and John Hewett
appeared half dressed.

``What's wrong?'' he cried. ``The `ouse o' fire? Who yelled out like
that?''

Clem was coming up; she spoke from the landing below.

{}``It's that Jane, just because I gave her a rap as she deserved. Send
her down again.''

``Oh, no!'' cried the poor girl. ``Miss Hewett! be a friend to me! She's
goin' to shut me up all night with the coffin. Don't let her, miss! I
durstn't! Oh, be a friend to me!''

``Little liar!'' shouted Clem. ``Oh, that bloomin little liar! when I
never said a word o' such a thing!''

``I'll believe her a good deal sooner than you,'' returned Clara
sharply. ``Why, anybody can see she's tellin' the truth,---can't they,
father? She's half scared out of her life. Come in here, Jane; you shall
stay here till morning.''

By this time all the grown-up people in the house were on the staircase;
the clang of tongues was terrific. Clem held her ground stoutly, and in
virulence was more than a match for all her opponents. Even Bob did not
venture to take her part; he grinned down over the banisters, and
enjoyed the entertainment immensely. Dick Snape, whose {}room Bob
shared, took the opportunity of paying off certain old scores he had
standing against Clem. Mr. Marple, the cab-driver, was very loud and
very hoarse in condemnation of such barbarity. Mrs. Hewett, looking as
if she had herself risen from a coffin, cried shame on the general
heartlessness with which Jane was used.

Clara held to her resolve. She led Jane into the bedroom, then, with a
parting shot at Miss Peckover, herself entered and locked the door.

``Drink some water, Jane,'' she said, doing her best to reassure the
child. ``You're safe for to-night, and we'll see what Mrs Peckover says
about this when she comes back to-morrow.''

Jane looked at her rescuer with eyes in which eternal gratitude mingled
with fear for the future. She could cry now, poor thing, and so little
by little recover herself. Words to utter her thanks she had none; she
could only look something of what she felt, Clara made her undress and
lie down with little {}Tom on the mattress. In a quarter of an hour the
candle was extinguished, and but for the wind, which rattled sashes and
doors, and made ghostly sounds in the chimneys, there was silence
throughout the house.

Something awoke Clara before dawn. She sat up, and became aware that
Jane was talking and crying wildly, evidently re-acting in her sleep the
scene of a few hours ago. With difficulty Clara broke her slumber.

``Don't you feel well, Jane?'' she asked, noticing a strangeness in the
child's way of replying to her.

``Not very, miss. My head's bad, an' I'm so thirsty. May I drink out of
the jug, miss?''

``Stay where you are. I'll bring it to you.''

Jane drank a great deal. Presently she fell again into slumber, which
was again broken in the same way. Clara did not go to sleep, and as soon
as it was daylight she summoned her father to come and look at the
child. Jane was ill, and, as every one could see, rapidly grew worse.

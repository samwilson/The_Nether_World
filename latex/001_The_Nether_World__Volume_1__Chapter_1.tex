\part{}

\chapter{A Thrall of Thralls}

\textsc{In} the troubled twilight of a March evening ten years ago, an
old man, whose equipment and bearing suggested that he was fresh from
travel, walked slowly across Clerkenwell Green, and by the graveyard of
St. James's Church stood for a moment looking about him. His age could
not be far from seventy, but, despite the stoop of his shoulders, he
gave little sign of failing under the burden of years; his sober step
indicated gravity of character rather than bodily feebleness, and his
grasp of a stout stick was not such as bespeaks need of support. His
attire was neither that of a man of leisure, nor of the kind usually
worn by English {}mechanics. Instead of coat and waistcoat, he wore a
garment something like a fisherman's guernsey, and over this a coarse
short cloak, picturesque in appearance as it was buffeted by the wind.
His trousers were of moleskin; his boots reached almost to his knees;
for head-covering he had the cheapest kind of undyed felt, its form
exactly that of the old petasus. To say that his aspect was venerable
would serve to present him in a measure, yet would not be wholly
accurate, for there was too much of past struggle and present anxiety in
his countenance to permit full expression of the natural dignity of the
features. It was a fine face, and might have been distinctly noble, but
circumstances had marred the purpose of Nature; you perceived that his
cares had too often been of the kind which are created by ignoble
necessities, such as leave to most men of his standing a bare humanity
of visage. He had long thin white hair; his beard was short and merely
grizzled. In his left hand he carried a bundle, which probably contained
clothing.

{}The burial-ground by which he had paused was as little restful to the
eye as are most of those discoverable in the byways of London. The small
trees that grew about it shivered in their leaflessness; the rank grass
was wan under the failing day; most of the stones leaned this way or
that, emblems of neglect, (they were very white at the top, and darkened
downwards till the damp soil made them black), and certain cats and dogs
were prowling or sporting among the graves. At this corner the east wind
blew with malice such as it never puts forth save where there are poorly
clad people to be pierced; it swept before it thin clouds of unsavoury
dust, mingled with the light refuse of the streets. Above the shapeless
houses night was signalling a murky approach; the sky---if sky it could
be called---gave threatening of sleet, perchance of snow. And on every
side was the rumble of traffic, the voiceful evidence of toil and of
poverty; hawkers were crying their goods; the inevitable organ was
clanging before a public-house hard by; {}the crumpet-man was hastening
along, with monotonous ringing of his bell and hoarse rhythmic wail.

The old man had fixed his eyes half absently on the inscription of a
gravestone near him; a lean cat springing out between the iron railings
seemed to recall his attention, and with a slight sigh he went forward
along the narrow street which is called St. James's Walk. In a few
minutes he had reached the end of it, and found himself facing a high
grey-brick wall, wherein, at this point, was an arched gateway closed
with black doors. He looked at the gateway, then fixed his gaze on
something that stood just above---something which the dusk half
concealed, and by so doing made more impressive. It was the sculptured
counterfeit of a human face, that of a man distraught with agony. The
eyes stared wildly from their sockets, the hair straggled in maniac
disorder, the forehead was wrung with torture, the cheeks sunken, the
throat fearsomely wasted, and from the wide lips there seemed to be
issuing a {}horrible cry. Above this hideous effigy was carved the
legend: \textsc{``Middlesex House of Detention''}.

Something more than pain came to the old man's face as he looked and
pondered; his lips trembled like those of one in anger, and his eyes had
a stern, resentful gleaming. He walked on a few paces, then suddenly
stopped where a woman was standing at an open door.

``I ask your pardon,'' he said, addressing her with the courtesy which
owes nothing to refined intercourse, ``but do you by chance know any one
of the name of Snowdon hereabouts?''

The woman replied with a brief negative; she smiled at the appearance of
the questioner, and, with the vulgar instinct, looked about for some one
to share her amusement.

``Better inquire at the `ouse at the corner,'' she added, as the man was
moving away. ``They've been here a long time, I b'lieve.''

He accepted her advice. But the people {}at the public-house could not
aid his search. He thanked them, paused for a moment with his eyes down,
then again sighed slightly and went forth into the gathering gloom.

Less than five minutes later there ran into the same house of
refreshment a little slight girl, perhaps thirteen years old; she
carried a jug, and at the bar asked for ``a pint of old six.'' The
barman, whilst drawing the ale, called out to a man who had entered
immediately after the child:

``Don't know nobody called Snowdon about 'ere, do you, Mr. Squibbs?''

The individual addressed was very dirty, very sleepy, and seemingly at
odds with mankind. He replied contemptuously with a word which, in
phonetic rendering, may perhaps be spelt ``Nay-oo.''

But the little girl was looking eagerly from one man to the other; what
had been said appeared to excite keen interest in her. She forgot all
about the beer-jug that was waiting, and, after a brief but obvious
struggle with timidity, said in an uncertain voice:

{}``Has somebody been asking for that name, sir?''

``Yes, they have,'' the barman answered, in surprise. ``Why?''

``My name's Snowdon, sir---Jane Snowdon.''

She reddened over all her face as soon as she had given utterance to the
impulsive words. The barman was regarding her with a sort of
semi-interest, and Mr. Squibbs also had fixed his bleary (or beery) eyes
upon her. Neither would have admitted an active interest in so pale and
thin and wretchedly-clad a little mortal. Her hair hung loose, and had
no covering; it was hair of no particular colour, and seemed to have
been for a long time utterly untended; the wind, on her run hither, had
tossed it into much disorder. Signs there were of some kind of clothing
beneath the short, dirty, worn dress, but it was evidently of the
scantiest description. The freely exposed neck was very thin, but, like
the outline of her face, spoke less of a feeble habit of body than of
the present pinch of sheer hunger. She did not, indeed, {}look like one
of those children who are born in disease and starvation and put to
nurse upon the pavement; her limbs were shapely enough, her back was
straight, she had features that were not merely human, but girl-like,
and her look had in it the light of an intelligence generally sought for
in vain among the children of the street. The blush and the way in which
she hung her head were likewise tokens of a nature endowed with ample
sensitiveness.

``Oh, your name's Jane Snowdon, is it?'' said the barman. ``Well, you're
just three minutes an' three-quarters too late. P'raps it's a fortune a
runnin' after you. He was a rum old party as inquired. Never mind; it's
all in a life. There's fortunes lost every week by a good deal less than
three minutes when it's 'orses---eh, Mr. Squibbs?''

Mr. Squibbs swore with emphasis.

The little girl took her jug of beer and was turning away.

``Hollo!'' cried the barman. "Where's the money, Jane?---if \emph{you}
don't mind."

{}She turned again in increased confusion, and laid coppers on the
counter. Thereupon the man asked her where she lived; she named a house
in Clerkenwell Close, near at hand.

``Father live there?''

She shook her head.

``Mother?''

``I haven't got one, sir.''

``Who is it as you live with, then?''

``Mrs. Peckover, sir.''

``Well, as I was sayin', he was a queer old joker as arsted for the name
of Snowdon. Shouldn't wonder if you see him goin' round.''

And he added a pretty full description of this old man, to which the
girl listened closely. Then she went thoughtfully---a little sadly---on
her way.

In the street, all but dark by this time, she cast anxious glances
onwards and behind, but no old man in an odd hat and cloak and with
white hair was discoverable. Linger she might not. She reached a house
of which the front-door stood open; it looked {}black and cavernous
within, but she advanced with the step of familiarity and went
downstairs to a front-kitchen. Through the half-open door came a strong
odour and a hissing sound, plainly due to the frying of sausages. Before
Jane could enter, she was greeted sharply in a voice which was young and
that of a female, but had no other quality of graciousness.

``You've taken your time, my lady! All right! just wait till I've `ad my
tea, that's all! Me an' you'll settle accounts to-night, see if we
don't. Mother told me as she owed you a lickin', an' I'll pay it off,
with a little on my own account too. Only wait till I've `ad my tea,
that's all. What are you standin' there for, like a fool? Bring that
beer `ere, an' let's see 'ow much you've drank.''

``I haven't put my lips near it, miss; indeed I haven't,'' pleaded the
child, whose face of dread proved both natural timidity and the constant
apprehension of ill-usage.

``Little liar! that's what you always was, an always will be.---Take
that!''

{}The speaker was a girl of sixteen, tall, rather bony, rudely handsome;
the hand with which she struck was large and coarse-fibred, the muscles
that impelled it vigorous. Her dress was that of a work-girl,
unsubstantial, ill-fitting, but of ambitious cut; her hair was very
abundant, and rose upon the back of her head in thick coils, an elegant
fringe depending in front. The fire had made her face scarlet, and in
the lamplight her large eyes glistened with many joys.

First and foremost, Miss Clementina Peckover rejoiced because she had
left work much earlier than usual, and was about to enjoy what she would
have described as a ``blow out.'' Secondly, she rejoiced because her
mother, the landlady of the house, was absent for the night, and
consequently she would exercise sole authority over the domestic slave,
Jane Snowdon, that is to say, would indulge to the uttermost her
instincts of cruelty in tormenting a defenceless creature. Finally---a
cause of happiness antecedent to {}the others, but less vivid in her
mind at this moment---in the next room lay awaiting burial the corpse of
Mrs. Peckover's mother-in-law, whose death six days ago had plunged
mother and daughter into profound delight, partly because they were
relieved at length from making pretence of humanity to a bed-ridden old
woman, partly owing to the fact that the deceased had left behind her a
sum of seventy-five pounds, exclusive of moneys due from a burial-club.

``Ah!'' exclaimed Miss Peckover (who was affectionately known to her
intimates as ``Clem''), as she watched Jane stagger back from the blow
and hide her face in silent endurance of pain. "That's just a morsel to
stay your appetite, my lady! You didn't expect me back `ome at this
time, did you? You thought as you was goin' to `ave the kitchen to
yourself when mother went. Ha ha! ho ho!---These sausages is done; now
you clean that fiyin'-pan; and if I can find a speck of dirt in it as
big as 'arf a farden, I'll take you by the 'air of {}the `ed an' clean
it with your face, \emph{that's} what I'll do! Understand? Oh, I mean
what I say, my lady! Me an' you's a-goin' to spend a evenin' together,
there's no two ways about that. Ho ho! he he!"

The frankness of Clem's brutality went far towards redeeming her
character. The exquisite satisfaction with which she viewed Jane's
present misery, the broad joviality with which she gloated over the
prospect of cruelties shortly to be inflicted, put her at once on a par
with the noble savage running wild in woods. Civilisation could bring no
charge against this young woman; it and she had no common criterion. Who
knows but this lust of hers for sanguinary domination was the natural
enough issue of the brutalising serfdom of her predecessors in the
family line of the Peckovers? A thrall suddenly endowed with authority
will assuredly make bitter work for the luckless creature in the next
degree of thraldom.

A cloth was already spread across one end of the deal table, with such
other {}preparations for a meal as Clem deemed adequate. The
sausages---five in number---she had emptied from the frying-pan directly
on to her plate, and with them all the black rich juice that had exuded
in the process of cooking---particularly rich, owing to its having
several times caught fire and blazed triumphantly. On sitting down and
squaring her comely frame to work, the first thing Clem did was to take
a long draught out of the beer-jug; refreshed thus, she poured the
remaining liquor into a glass. Ready at hand was mustard, made in a
tea-cup; having taken a certain quantity of this condiment on to her
knife, she proceeded to spread each sausage with it from end to end,
patting them in a friendly way as she finished the operation. Next she
sprinkled them with pepper, and after that she constructed a little pile
of salt on the side of the plate, using her fingers to convey it from
the salt-cellar. It remained to cut a thick slice of bread---she held
the loaf pressed to her bosom whilst doing this---and {}to crush it down
well into the black grease beside the sausages; then Clem was ready to
begin.

For five minutes she fed heartily, showing really remarkable skill in
conveying pieces of sausage to her mouth by means of the knife alone.
Finding it necessary to breathe at last, she looked round at Jane. The
hand-maiden was on her knees near the fire, scrubbing very hard at the
pan with successive pieces of newspaper. It was a sight to increase the
gusto of Clem's meal, but of a sudden there came into the girl's mind a
yet more delightful thought. I have mentioned that in the back-kitchen
lay the body of a dead woman; it was already encoffined, and waited for
interment on the morrow, when Mrs. Peckover would arrive with a certain
female relative from St. Albans. Now the proximity of this corpse was a
ceaseless occasion of dread and misery to Jane Snowdon; the poor child
had each night to make up a bed for herself in this front-room, dragging
together a little heap of rags when {}mother and daughter were gone up
to their chamber, and since the old woman's death it was much if Jane
had enjoyed one hour of unbroken sleep. She endeavoured to hide these
feelings, but Clem, with her Red Indian scent, divined them accurately
enough. She hit upon a good idea.

``Go into the next room,'' she commanded suddenly, ``and fetch the
matches off of the mantelpiece. I shall want to go upstairs presently,
to see if you've scrubbed the bedroom well.''

Jane was blanched; but she rose from her knees at once, and reached a
candlestick from above the fireplace. ``What's that for?'' shouted Clem,
with her mouth full. ``You've no need of a light to find the
mantelpiece. If you're not off{{------}}"

Jane hastened from the kitchen. Clem yelled to her to close the door,
and she had no choice but to obey. In the dark passage outside there was
darkness that might be felt. The child all but fainted with the
{}sickness of horror as she turned the handle of the other door and
began to grope her way. She knew exactly where the coffin was; she knew
that to avoid touching it in the diminutive room was all but impossible.
And touch it she did. Her anguish uttered itself, not in a mere sound of
terror, but in a broken word or two of a prayer she knew by heart,
including a name which sounded like a charm against evil. She had
reached the mantelpiece; oh, she could not, could not find the matches!
Yes, at last her hand closed on them. A blind rush, and she was out
again in the passage. She re-entered the front-kitchen with limbs that
quivered, with the sound of dreadful voices ringing about her, and
blankness before her eyes.

Clem laughed heartily, then finished her beer in a long, enjoyable pull.
Her appetite was satisfied; the last trace of oleaginous matter had
disappeared from her plate, and now she toyed with little pieces of
bread lightly dipped into the mustard-pot. These {}\emph{bonnes bouches}
put her into excellent humour; presently she crossed her arms and leaned
back. There was no denying that Clem was handsome; at sixteen she had
all her charms in apparent maturity, and they were of the coarsely
magnificent order. Her forehead was low and of great width; her nose was
well shapen, and had large sensual apertures; her cruel lips may be seen
on certain fine antique busts; the neck that supported her heavy head
was splendidly rounded. In laughing, she became a model for an artist,
an embodiment of fierce life independent of morality. Her health was
probably less sound than it seemed to be; one would have compared her,
not to some piece of exuberant normal vegetation, but rather to a rank,
evilly-fostered growth. The putrid soil of that nether world yields
other forms besides the obviously blighted and sapless.

``Have you done any work for Mrs. Hewett to-day?'' she asked of her
victim, after sufficiently savouring the spectacle of terror.

{}``Yes, miss; I did the front-room fireplace, an' fetched fourteen of
coals, an' washed out a few things.''

``What did she give you?''

``A penny, miss. I gave it to Mrs. Peckover before she went.''

``Oh, you did? Well, look `ere; you'll just remember in future that all
you get from the lodgers belongs to me, an' not to mother. It's a new
arrangement, understand. An' if you dare to give up a `apenny to mother,
I'll lick you till you're nothin' but a bag o' bones. Understand?''

Having on the spur of the moment devised this ingenious difficulty for
the child, who was sure to suffer in many ways from such a conflict of
authorities, Clem began to consider how she should spend her evening.
After all, Jane was too poor-spirited a victim to afford long
entertainment. Clem would have liked dealing with some one who showed
fight---some one with whom she could try savage issue in real
tooth-and-claw conflict. She had in mind a really exquisite piece {}of
cruelty, but it was a joy necessarily postponed to a late hour of the
night. In the meantime, it would perhaps be as well to take a stroll,
with a view of meeting a few friends as they came away from the
work-rooms. She was pondering the invention of some long and hard task
to be executed by Jane in her absence, when a knocking at the house-door
made itself heard. Clem at once went up to see who the visitor was.

A woman in a long cloak and a showy bonnet stood on the step, protecting
herself with an umbrella from the bitter sleet which the wind was now
driving through the darkness. She said that she wished to see Mrs.
Hewett.

``Second-floor front,'' replied Clem in the offhand, impertinent tone
wherewith she always signified to strangers her position in the house.

The visitor regarded her with a look of lofty contempt, and, having
deliberately closed her umbrella, advanced towards the stairs. Clem drew
into the back regions for {}a few moments, but as soon as she heard the
closing of a door in the upper part of the house, she too ascended,
going on tip-toe, with a noiselessness which indicated another side of
her character. Having reached the room which the visitor had entered,
she brought her ear close to the keyhole, and remained in that attitude
for a long time---nearly twenty minutes, in fact. Her sudden and swift
return to the foot of the stairs was followed by the descent of the
woman in the showy bonnet.

``Miss Peckover!'' cried the latter when she had reached the foot of the
stairs.

``Well, what is it?'' asked Clem, seeming to come up from the kitchen.

``Will you 'ave the goodness to go an speak to Mrs. Hewett for a
hinstant?'' said the woman, with much affectation of refined speech.

``All right! I will just now, if I've time.'' The visitor tossed her
head and departed, whereupon Clem at once ran upstairs. In five minutes
she was back in the kitchen.

{}``See 'ere,'' she addressed Jane. ``You know where Mr. Kirkwood works
in St. John's Square? You've been before now. Well, you're to go an'
wait at the door till he comes out, and then you're to tell him to come
to Mrs. Hewett at wunst. Understand?---Why ain't these tea-things all
cleared away? All right! Wait till you come back, that's all. Now be
off, before I skin you alive!''

On the floor in a corner of the kitchen lay something that had once been
a girl's hat. This Jane at once snatched up and put on her head. Without
other covering, she ran forth upon her errand.

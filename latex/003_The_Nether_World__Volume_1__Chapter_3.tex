\chapter{A Superfluous Family}

\textsc{Kirkwood's} face, as he turned to greet the new-comer, changed
suddenly to an expression of surprise.

``Why, what have you been doing to your hair?'' he asked abruptly.

A stranger would have seen nothing remarkable in John Hewett's hair,
unless he had reflected that, being so sparse, it had preserved its dark
hue and its gloss somewhat unusually. The short beard and whiskers were
also of richer colour than comported with the rest of the man's
appearance. Judging from his features alone, one would have taken John
for sixty at least; his years were in truth not quite two-and-fifty. He
had the look of one worn out with anxiety {}and hardship; the lines
engraven upon his face were of extraordinary depth and frequency; there
seemed to he little flesh between the dry skin and the bones which
sharply outlined his visage. The lips were, like those of his son,
prominent and nervous, but none of Bob's shrewdness was here
discoverable; feeling rather than intellect appeared to be the father's
characteristic. His eyes expressed self-will, perhaps obstinacy, and he
had a peculiarly dogged manner of holding his head. At the present
moment he was suffering from extreme fatigue; he let himself sink upon a
chair threw his hat on to the floor, and rested a hand on each knee. His
boots were thickly covered with mud; his corduroy trousers were splashed
with the same. Rain had drenched him; it trickled to the floor from all
his garments.

For answer to Sidney's question, he nodded towards his wife and said in
a thick voice, ``Ask her.''

``He's dyed it,'' Mrs. Hewett explained, with no smile. "He thought one
of the {}reasons why he couldn't get work was his lookin' too old."

``An' so it was,'' exclaimed Hewett, with an angry vehemence which at
once declared his position and revealed much of his history. ``So it
was! My hair was a bit turned, an' now-a-days there's no chance for old
men. Ask any one you like. Why, there's Sam Lang couldn't even get a job
at gardenin' `cause his hair was a bit turned. It was him as told me
what to do. 'Dye your hair, Jack,' he says; `it's what I've had to
myself,' he says. `They won't have old men now-adays, at no price.' Why,
there's Jarvey the painter; you know him, Sidney. His guvnor sent him on
a job to Jones's place, an' they sent him back. `Why, he's an old man,'
they says. `What good's a man of that age for liftin' ladders about?'
An' Jarvey's no older than me.''

Sidney knitted his brows. He had heard the complaint from too many men
to be able to dispute its justice.

"When there's twice too many of us for {}the work that's to be done,"
pursued John, ``what else can you expect? The old uns have to give way,
of course. Let 'em beg; let 'em starve! What use are they?''

Mrs. Hewett had put a kettle on the fire, and began to arrange the table
for a meal.

``Go an' get your wet things off, John,'' she said. ``You'll be havin'
your rheumatics again.''

``Never mind me, Maggie. What business have you to be up an' about? You
need a good deal more takin' care of than I do. Here, let Amy get the
tea.''

The three children. Amy, Annie, and Tom, had come forward, as only
children do who are wont to be treated affectionately on their father's
return. John had a kiss and a caress for each of them; then he stepped
to the bed and looked at his latest born. The baby was moaning feebly;
he spoke no word to it, and on turning away glanced about the room
absently. In the meantime his wife had taken some clothing from a chest
of drawers, and at length he was persuaded to {}go into the other room
and change. When he returned, the meal was ready. It consisted of a
scrap of cold steak, left over from yesterday, and still upon the
original dish amid congealed fat; a spongy half-quartern loaf, that
species of baker's bread of which a great quantity can be consumed with
small effect on the appetite; a shapeless piece of something purchased
under the name of butter, dabbed into a shallow basin; some pickled
cabbage in a tea-cup; and lastly, a pot of tea, made by adding a
teaspoonful or two to the saturated leaves which had already served at
breakfast and mid-day. This repast was laid on a very dirty cloth. The
cups were unmatched and chipped, the knives were in all stages of
decrepitude; the teapot was of dirty tin, with a damaged spout.

Sidney began to affect cheerfulness. He took little Annie on one of his
knees, and Tom on the other. The mature Amy presided. Hewett ate the
morsel of meat, evidently without thinking about it; he crumbled a piece
of bread, and munched mouthfuls in {}silence. Of the vapid liquor called
tea lie drank cup after cup.

``What's the time? '' he asked at length.

``Where's Clara?''

``I daresay she's doin' overtime,'' replied his wife. ``She won't be
much longer.''

The man was incapable of remaining in one spot for more than a few
minutes. Now he went to look at the baby; now he stirred the fire; now
he walked across the room aimlessly. He was the embodiment of worry. As
soon as the meal was over, Amy, Annie, and Tom were sent off to bed.
They occupied the second room, together with Clara; Bob shared the bed
of a fellow-workman upstairs. This was great extravagance, obviously;
other people would have made two rooms sufficient for all, and many such
families would have put up with one. But Hewett had his ideas of
decency, and stuck to them with characteristic wilfulness.

``Where do you think I've been this afternoon?'' John began, when the
three little ones were gone, and Mrs. Hewett had {}been persuaded to lie
down upon the bed. "Walked to Enfield an' back. I was told of a job out
there; but it's no good; they're full up. They say exercise is good for
the 'ealth. I shall be a 'ealthy man before long, it seems to me. What
do \emph{you} think?"

``Have you been to see Corder again? '' asked Sidney, after reflecting
anxiously.

``No, I haven't!'' was the angry reply; "an' what's more, I ain't goin'
to! He's one o' them men I can't get on with. As long as you make
yourself small before him, an' say ` sir ' to him with every other word,
an' keep tellin' him as he's your Providence on earth, an' as you don't
know how ever you'd get on without him---well, it's all square, an'
he'll keep you on the job. That's just what I \emph{can't} do---never
could, an' never shall. I should have to hear them children cryin for
food before I could do it. So don't speak to me about Corder again. It
makes me wild!"

Sidney tapped the floor with his foot. Himself a single man, without
responsibilities, always in fairly good work, he could not {}invariably
sympathise with Hewett's sore and impracticable pride. His own temper
did not err in the direction of meekness, but as he looked round the
room, he felt that a home such as this would drive him to any degree of
humiliation. John knew what the young man's thoughts were; he resumed in
a voice of exasperated bitterness.

"No, I haven't been to Corder,---I beg his pardon; \emph{Mister}
Corder,---James Corder, Esquire. But where do you think I went this
mornin'? Mrs. Peckover brought up a paper an' showed me an
advertisement. Gorbutt in Goswell Road wanted a man to clean windows an'
sweep up, an' so on;---offered fifteen bob a week. Well, I went. Didn't
I, mother? Didn't I go after that job? I got there at half-past eight;
an' what do you think I found? If there was one man standin' at
Gorbutt's door, \emph{there was five hundred!} Don't you believe me? You
go an' ask them as lives about there. If there was one, there was five
hundred! Why, the p'lice had to come an' keep the {}road clear. Fifteen
bob! What was the use o' me standin' there, outside the crowd? What was
the use, I say? Such a lot o' poor starvin' devils you never saw brought
together in all your life. There they was, lookin' ready to fight with
one another for the fifteen bob a week. Didn't I come back an' tell you
about it, mother? An' if they'd all felt like me, they'd a turned
against the shop an' smashed it up,---ay, an' every other shop in the
street! What use? Why, no use; but I tell you that's how I felt. If any
man had said as much as a rough word to me, I'd a gone at him like a
bulldog. I felt like a beast. I wanted to fight, I tell you,---to fight
till the life was kicked an' throttled out of me!"

``John, don't, don't go on in that way,'' cried his wife, sobbing
miserably. ``Don't let him go on like that, Sidney.''

Hewett jumped up and walked about.

``What's the time?'' he asked the next moment. And when Sidney told him
that it was half-past nine, he exclaimed, "Then {}why hasn't Clara come
'ome? What's gone with her?"

``Perhaps she's at Mrs. Tubbs's,'' replied his wife, in a low voice,
looking at Kirkwood.

``An' what call has she to be there? Who gave her leave to go there?''

There was another exchange of looks between Sidney and Mrs. Hewett; then
the latter with hesitation and timidity, told of Mrs. Tubbs's visit to
her that evening, and of the proposals the woman had made.

``I won't hear of it!'' cried John. ``I won't have my girl go for a
barmaid, so there's an end of it. I tell you she shan't go!''

``I can understand you, Mr. Hewett,'' said Sidney, in a tone of argument
softened by deference; ``but don't you think you'd better make a few
inquiries, at all events. You see, it isn't exactly a barmaid's place. I
mean to say, Mrs. Tubbs doesn't keep a public-house where people stand
about drinking all day. It is only a luncheon-bar, and respectable
enough.''

{}John turned and regarded him with astonishment.

``Why, I thought you was as much set against it as me? What's made you
come round like this? I s'pose you've got tired of her, an' that's made
you so you don't care.''

The young man's eyes flashed angrily, but before he could make a
rejoinder Mrs. Hewett interposed.

``For shame o' yourself, John! If you can't talk better sense than that,
don't talk at all. He don't mean it, Sidney. He's half drove off his
head with trouble.''

``If he does think it,'' said Kirkwood, speaking sternly but with
self-command, ``let him say what he likes. He can't say worse than I
should deserve.''

There was an instant of silence. Hewett's head hung with more than the
usual doggedness. Then he addressed Sidney, sullenly, but in a tone
which admitted his error.

"What have you got to say? Never mind me. I'm only the girl's father,
an' there's not {}much heed paid to fathers now-a-days. What have you
got to say about Clara? If you've changed your mind about her goin'
there, just tell me why."

Sidney could not bring himself to speak at once, but an appealing look
from Mrs. Hewett decided him.

``Look here, Mr. Hewett,'' he began, with blunt earnestness. ``If any
harm came to Clara, I should feel it every bit as much as you, and that
you ought to know by this time. All the same, what I've got to say is
this: Let her go to Mrs. Tubbs for a month's trial. If you persist in
refusing her, mark my words, you'll be sony. I've thought it all over,
and I know what I'm talking about. The girl can't put up with the
workroom any longer. It's ruining her health, for one thing, anybody can
see that, and it's making her so discontented, she'll soon get reckless.
I understand your feeling well enough, but I understand her as well; at
all events, I believe I do. She wants a change; she's getting tired of
her very life.''

{}``Very well,'' cried the father in shrill irritation, ``why doesn't
she take the change that's offered to her? She's no need to go neither
to workroom nor to bar. There's a good home waiting for her, isn't
there? What's come to the girl? She used to go on as if she liked you
well enough.''

``A girl alters a deal between fifteen and seventeen,'' Sidney replied,
forcing himself to speak with an air of calmness, of impartiality. "She
wasn't old enough to know her own mind. I'm tired of plaguing her. I
feel ashamed to say another word to her, and that's the truth. She only
gets more and more set against me. If it's ever to come right, it'll
have to be by waiting; we won't talk about that any more. Think of her
quite apart from me, and what I've been hoping. She's seventeen years
old. You can't deal with a girl of that age like you can with Amy and
Annie. You'll have to trust her, Mr. Hewett. You'll have to, because
there's no help for it. We're working people, we are; we're the lower
orders; our {}girls have to go out and get their livings. We teach them
the best we can, and the devil knows they've got examples enough of
misery and ruin before their eyes to help them to keep straight. Rich
people can take care of their daughters as much as they like; they can
treat them like children till they're married; people of our kind can't
do that, and it has to be faced."

John sat with dark brow, his eyes staring on vacancy.

``It's right what Sidney says, father,'' put in Mrs. Hewett; ``we can't
help it.''

``You may perhaps have done harm when you meant only to do good,''
pursued Sidney.

"Always being so anxious, and showing what account you make of her,
perhaps you've led her to think a little too much of herself. She knows
other fathers don't go on in that way. And now she wants more freedom,
she feels it worse than other girls do when you begin to deny her. Talk
to her in a different way; talk as if you trusted her. Depend upon it,
it's the only hold you have upon her. Don't {}be so much afraid. Clara
has her faults,---I see them as well as any one,---but I'll never
believe she'd darken your life of her own free will."

There was an unevenness, a jerky vehemence, in his voice, which told how
difficult it was for him to take this side in argument. He often
hesitated, obviously seeking phrases which should do least injury to the
father's feelings. The expression of pain on his forehead and about his
lips testified to the sincerity with which he urged his views, at the
same time to a lurking fear lest impulse should be misleading him.
Hewett kept silence, in aspect as far as ever from yielding. Of a sudden
he raised his hand, and said, ``Husht!'' There was a familiar step on
the stairs. Then the door opened and admitted Clara.

The girl could not but be aware that the conversation she interrupted
had reference to herself Her father gazed fixedly at her; Sidney glanced
towards her with self-consciousness, and at once averted his eyes;
{}Mrs. Hewett examined her with apprehension. Having carelessly closed
the door with a push, she placed her umbrella in the corner and began to
unbutton her gloves. Her attitude was one of affected unconcern; she
held her head stiffly, and let her eyes wander to the farther end of the
room. The expression of her face was cold, preoccupied; she bit her
lower lip so that the under part of it protruded.

``Where have you been, Clara?'' her father asked.

She did not answer immediately, but finished drawing off her gloves and
rolled them up by turning one over the other. Then she said
indifferently:

``I've been to see Mrs. Tubbs.''

``And who gave you leave?'' asked Hewett with irritation.

``I don't see that I needed any leave. I knew she was coming here to
speak to you or mother, so I went, after work, to ask what you'd said.''

She was not above the middle stature of {}women, but her slimness and
erectness, and the kind of costume she wore, made her seem tall as she
stood in this low-ceiled room. Her features were of very uncommon type,
at once sensually attractive and bearing the stamp of intellectual
vigour. The profile was cold, subtle, original; in full face, her high
cheek-bones and the heavy, almost horizontal line of her eyebrows were
the points that first drew attention, conveying an idea of force of
character. The eyes themselves were hazel-coloured, and, whatever her
mood, preserved a singular pathos of expression, a look as of self-pity,
of unconscious appeal against some injustice. In contrast with this, her
lips were defiant, insolent, unscrupulous; a shadow of the naïveté of
childhood still lingered upon them, but, though you divined the earlier
pout of the spoilt girl, you felt that it must have foretold this
danger-signal in the mature woman. Such cast of countenance could belong
only to one who intensified in her personality an inheritance of revolt;
who, combining the temper of an ambitious woman {}with the forces of a
man's brain, had early learnt that the world was not her friend nor the
world's law.

Her clothing made but poor protection against the rigours of a London
winter. Its peculiarity (bearing in mind her position) was the lack of
any pretended elegance. A close-fitting, short jacket of plain cloth
made evident the grace of her bust; beneath was a brown dress, with one
row of kilting. She wore a hat of brown felt, the crown rising from back
to front, the narrow brim closely turned up all round. The high collar
of the jacket alone sheltered her neck. Her gloves, though worn, were
obviously of good kid; her boots---strangest thing of all in a
work-girl's daily attire---were both strong and shapely. This simplicity
seemed a declaration that she could not afford genuine luxuries and
scorned to deck herself with shams.

The manner of her reply inflamed Hewitt with impotent wrath. He smote
the table violently, then sprang up and flung his chair aside.

{}``Is that the way you've learnt to speak to your father?'' he shouted.
``Haven't I told you you're not to go nowhere without my leave or your
mother's? Do you pay no heed to what I bid you? If so, say it! Say it at
once, and have done with it.''

Clara was quietly removing her hat. In doing so, she disclosed the one
thing which gave proof of regard for personal appearance. Her hair was
elaborately dressed. Drawn up from the neck, it was disposed in thick
plaits upon the top of her head; in front were a few rows of crisping.
She affected to be quite unaware that words had been spoken to her, and
stood smoothing each side of her forehead.

John strode forward and laid his hands roughly upon her shoulders.

``Look at me, will you? Speak, will you?''

Clara jerked herself from his grasp and regarded him with insolent
surprise. Of fear there was no trace upon her countenance; she seemed to
experience only astonishment {}at such unwonted behaviour from her
father, and resentment on her own behalf. Sidney Kirkwood had risen and
advanced a step or two, as if in apprehension of harm to the girl, but
his interference was unneeded. Hewett recovered his self-control as soon
as Clara repelled him. It was the first time he had ever laid a hand
upon one of his children other than gently; his exasperation came of
over-tried nerves, of the experiences he had gone through in search of
work that day, and the keen suffering occasioned by his argument with
Sidney. The practical confirmation of Sidney's warning that he must no
longer hope to control Clara like a child stung him too poignantly; he
obeyed an unreasoning impulse to recover his authority by force.

The girl's look entered his heart like a stab; she had never faced him
like this before, saying more plainly than with words that she defied
him to control her. His child's face, the face he loved best of all! yet
at this moment he was searching it vainly {}for the lineaments that were
familiar to him. Something had changed her, had hardened her against
him, in a moment. It seemed impossible that there should come such
severance between them. John revolted against it, as against all the
other natural laws that visited him harshly.

``What's come to you, my girl?'' he said in a thick voice. ``What's
wrong between us, Clara? Haven't I always done my best for you? If I was
the worst enemy you had, you couldn't look at me crueller.''

"I think it's me that should ask what's come to \emph{you}, father," she
returned, with her former self-possession. ``You treat me as if I was a
baby.---I want to know what you're going to say about Mrs. Tubbs. I
suppose mother's told you what she offers me?''

Sidney had not resumed his chair. Before Hewett could reply he said:

``I think I'll leave you to talk over this alone.''

``No; stay where you are,'' said John {}gruffly. ``Look here, Clara.
Sidney's been talkin' to me; he's been sayin' that I ought to let you
have your own way in this. Yes, you may well look as if it surprised
you.'' Clara had just glanced at the young man, slightly raising her
eyebrows, but at once looked away again with a careless movement of the
head. "He says what it's hard an' cruel for me to believe, though I half
begin to see that he's right; he says you won't pay no more heed to what
\emph{I} wish, an' it's me now must give way to you. I didn't use to
think me an' Clara would come to that; but it looks like it---it looks
like it."

The girl stood with downcast eyes. Once more her face had suffered a
change; the lips were no longer malignant, her forehead had relaxed from
its haughty frown. The past fortnight had been a period of contest
between her father's stubborn fears and her own determination to change
the mode of her life. Her self-will was only intensified by opposition.
John had often enough experienced this, but hitherto the points at
{}issue had been trifles, matters in which the father could yield for
the sake of pleasuring his child. Serious resistance brought out for the
first time all the selfish forces of her nature. She was prepared to go
all lengths rather than submit, now the question of her liberty had once
been broached. Already there was a plan in her mind for quitting home,
regardless of all the misery she would cause, reckless of what future
might be in store for herself. But the first sign of yielding on her
father's part touched the gentler elements of her nature. Thus was she
constituted; merciless in egotism when put to the use of all her
weapons, moved to warmest gratitude as soon as concession was made to
her. To be on ill terms with her father had caused her pain, the only
effect of which, however, was to heighten the sullen impracticability of
her temper. At the first glimpse of relief from overstrained emotions,
she desired that all angry feeling should be at an end. Having gained
her point, she could once more be the {}affectionately wilful girl whose
love was the first necessity of John Hewett's existence.

``Well,'' John pursued, reading her features eagerly, ``I'll say no more
about that, and I won't stand in the way of what you've set your mind
on. But understand, Clara, my girl! It's because Sidney persuaded me.
Sidney answers for it, mind you that!''

His voice trembled, and he looked at the young man with something like
anger in his eyes.

``I'm willing to do that, Mr. Hewett,'' said Kirkwood in a low but firm
voice, his eyes turned away from Clara. ``No human being can answer for
another in the real meaning of the word; but I take upon myself to say
that Clara will bring you no sorrow. She hears me say it. They're not
the kind of words that a man speaks without thought of what they mean.''

Clara had seated herself by the table, and was moving a finger along the
pattern of the dirty white cloth. She bit her under-lip in the manner
abeady described, seemingly {}her habit when she wished to avoid any
marked expression of countenance.

``I can't see what Mr. Kirkwood's got to do with it at all,'' she said
with indifference, which now, however, was rather good-humoured than the
reverse. "I'm sure I don't want anybody to answer for \emph{me}" A
slight toss of the head. ``You'd have let me go in any case, father; so
I don't see you need bring Mr. Kirkwood's name in.''

Hewett turned away to the fireplace and hung his head. Sidney, gazing
darkly at the girl, saw her look towards him, and she smiled. The
strange effect of that smile upon her features! It gave gentleness to
the mouth, and, by making more manifest the intelligent light of her
eyes, emphasised the singular pathos inseparable from their regard. It
was a smile to which a man would concede anything, which would vanquish
every prepossession, which would inspire pity and tenderness and
devotion in the heart of sternest resentment.

Sidney knew its power only too well; he {}averted his face. Then Clara
rose again and said:

``I shall just walk round and tell Mrs. Tubbs. It isn't late, and she'd
like to know as soon as possible.''

``Oh, surely it'll do in the mornin'! '' exclaimed Mrs. Hewett, who had
followed the conversation in silent anxiety. Clara paid no attention,
but at once put on her hat again. Then she said, ``I won't be long,
father,'' and moved towards the door.

Hewett did not look round.

``Will you let me walk part of the way with you?'' Sydney asked
abruptly.

``Certainly, if you like.''

He bade the two who remained ``Goodnight,'' and followed Clara
downstairs.

\chapter{Clara's Return}

\textsc{Mrs. Eagles}, a middle-aged woman of something more than average
girth, always took her time in ascending to that fifth storey where she
and her husband shared a tenement with the Hewett family. This afternoon
her pause on each landing was longer than usual, for a yellow fog, which
mocked the pale glimmer of gas-jets on the staircase, made her gasp
asthmatically. She carried, too, a heavy market-bag, having done her
Saturday purchasing earlier than of wont on account of the intolerable
weather. She reached the door at length, and being too much exhausted to
search her pocket for the latch-key, knocked for admission. Amy Hewett
opened to her, and she sank on a chair in the first room, where the
other two Hewett children were bendins: over ``home-lessons'' with a
{}studiousness not altogether natural. Mrs. Eagles had a shrewd eye;
having glanced at Annie and Tom with a discreet smile, she turned her
look towards the elder girl, who was standing full in the lamplight.

``Come here, Amy,'' she said, after a moment's scrutiny. ``So you
\emph{will} keep doin' that foolish thing! Very well, then, I shall have
to speak to your father about it; I'm not goin' to see you make yourself
ill and do nothing to prevent you.''

Amy, now a girl of eleven, affected much indignation.

``Why, I haven't touched a drop, Mrs. Eagles!''

``Now, now, now, now, now! Why, your lips are shrivelled up like a bit
o' dried orange-peel! You're a silly girl, that's what you are!''

Of late Amy Hewett had become the victim of a singular propensity;
whenever she could obtain vinegar, she drank it as a toper does spirits.
Inadequate nourishment, and especially an unsatisfied palate, frequently
have this result in female children among the poor; {}it is an
anticipation of what will befall them as soon as they find their way to
the public-house.

Having administered a scolding, Mrs. Eagles went into the room which she
and her husband occupied. It was so encumbered with furniture that not
more than eight or ten square feet of floor can have been available for
movement. On the bed sat Mr. Eagles, a spare, large-headed, ugly, but
very thoughtful-looking man; he and Sidney Kirkwood had been
acquaintances and fellow-workmen for some years, but no close intimacy
had arisen between them, owing to the difference of their tastes and
views. Eagles was absorbed in the study of a certain branch of political
statistics; the enthusiasm of his life was Financial Reform. Every
budget presented to Parliament he criticised with extraordinary
thoroughness, and, in fact, with an acumen which would have made him no
inefficient auxiliary of the Chancellor himself. Of course he took the
view that the nation's resources were iniquitously wasted, and of course
had little difficulty in illustrating a truth so {}obvious; what
distinguished him from the ordinary malcontent of Clerkenwell Green was
his logical faculty and the surprising extent of the information with
which he had furnished himself. Long before there existed a ``Financial
Reform Almanack,'' Eagles practically represented that work in his own
person. Disinterested, ardent, with thoughts for but one subject in the
scope of human inquiry, he lived contentedly on his two pounds a week,
and was for ever engaged in the theoretic manipulation of millions.
Utopian budgets multiplied themselves in his brain and his note-books.
He devised imposts such as Minister never dreamt of, yet which, he
declared, could not fail of vast success. ``You just look at these
figures!'' he would exclaim to Sidney, in his low, intense voice.
``There it is in black and white!'' But Sidney's faculties were quite
unequal to calculations of this kind, and Eagles could never summon
resolve to explain his schemes before an audience. Indefatigably he
worked on, and the work had to be its own reward.

{}He was busy in the usual way this afternoon, as he sat on the bed,
coatless, a trade journal open on his knees. His wife never disturbed
him; she was a placid, ruminative woman, generally finding the details
of her own weekly budget quite a sufficient occupation. When she had
taken off her bonnet and was turning out the contents of her bag, Eagles
remarked quietly:

``They'll have a bad journey.''

``What a day for her to be travelling all that distance, poor thing! But
perhaps it ain't so bad out o' London.''

Lowering their voices, they began to talk of John Hewett and the
daughter he was bringing from Lancashire, where she had lain in hospital
for some weeks. Of the girl and her past they knew next to nothing, but
Hewett's restricted confidences suggested disagreeable things. The truth
of the situation was, that John had received by post, from he knew not
whom, a newspaper report of the inquest held on the body of Grace
Danver, wherein, of course, was an account of what {}had happened to
Clara Vale; in the margin was pencilled, ``Clara Vale's real name is
Clara Hewett.'' An hour after receiving this John encountered Sidney
Kirkwood. They read the report together. Before the coroner it had been
made public that the dead woman was in truth named Rudd; she who was
injured refused to give any details concerning herself, and her history
escaped the reporters. Harbouring no doubt of the information thus
mysteriously sent him,---the handwriting seemed to be that of a man, but
gave no further hint as to its origin,---Hewett the next day journeyed
down into Lancashire, Sidney supplying him with money. He found Clara in
a perilous condition; her face was horribly burnt with vitriol, and the
doctors could not as yet answer for the results of the shock she had
suffered. One consolation alone offered itself in the course of Hewett's
inquiries: Clara, if she recovered, would not have lost her eyesight.
The fluid had been thrown too low to effect the worst injury; the
accident of a trembling hand, of a movement on her part, had kept her
eyes untouched.

{}Necessity brought the father back to London almost at once, but the
news sent him at brief intervals continued to be favourable. Now that
the girl could be removed from the infirmary, there was no retreat for
her but her fathers home. Mr, Peel, the manager, had made her a present
of £20---it was all he could do; the members of the company had
subscribed another £5, generously enough, seeing that their tour was
come perforce to an abrupt close. Clara's career as an actress had
ended{.~.~.~.}

When the fog's artificial night deepened at the close of the winter
evening, Mrs. Eagles made the Hewetts' two rooms as cheerful as might
be, expecting every moment the arrival of John and his companion. The
children were aware that an all but forgotten sister was returning to
them, and that she had been very ill; they promised quietude. Amy set
the tea-table in order, and kept the kettle ready{.~.~.~.} The knock for
which they were waiting! Mrs. Eagles withdrew into her own room; Amy
went to the door.

{}A tall figure, so wrapped and veiled that nothing but the womanly
outline could be discerned, entered, supported by John Hewett.

``Is there a light in the other room. Amy?''

John inquired in a thick voice.

``Yes, father.''

He led the muffled form into the chamber where Amy and Annie slept. The
door closed, and for several minutes the three children stood regarding
each other, alarmed, mute. Then their father joined them. He looked
about in an absent way, slowly drew off his overcoat, and when Amy
offered to take it, bent and kissed her cheek. The girl was startled to
hear him sob and to see tears starting from his eyes. Turning suddenly
away, he stood before the fire and made a pretence of warming himself;
but his sobs overmastered him. He leaned his arms on the mantelpiece.

``Shall I pour out the tea, father?'' Amy ventured to ask, when there
was again perfect silence.

{}``Haven't you had yours?'' he replied, half-facing her.

``Not yet.''

``Get it, then,---all of you. Yes, you can pour me out a cup,---and put
another on the little tray. Is this stuff in the saucepan ready?''

``Mrs. Eagles said it would be in five minutes.''

``All right. Get on with your eatin', all of you.''

He went to Mrs. Eagles' room and talked there for a short time.
Presently Mrs. Eagles herself came out and silently removed from the
saucepan a mixture of broth and meat. Having already taken the cup of
tea to Clara, Hewett now returned to her with this food. She was sitting
by the fire, her face resting upon her hands. The lamp was extinguished;
she had said that the firelight was enough. John deposited his burden on
the table, then touched her shoulder gently and spoke in so soft a voice
that one would not have recognised it as his.

{}``You'll try an' eat a little, my dear? Here's somethin' as has been
made particular. After travellin',---just a spoonful or two.''

Clara expressed reluctance.

``I don't feel hungry, father. Presently, perhaps.''

``Well, well; it do want to cool a bit. Do you feel able to sit up?''

``Yes. Don't take so much trouble, father. I'd rather you left me
alone.''

The tone was not exactly impatient; it spoke a weary indifference to
everything and every person.

``Yes, I'll go away, dear. But you'll eat just a bit? If you don't like
this, you must tell me, and I'll get something you could fancy.''

``It'll do well enough. I'll eat it presently; I promise you.''

John hesitated before going.

``Clara,---shall you mind Amy and Annie comin' to sleep here? If you'd
rather, we'll manage it somehow else.''

``No. What does it matter? They can {}come when they like, only they
mustn't want me to talk to them.''

He went softly from the room, and joined the children at their tea. His
mood had grown brighter. Though in talking he kept his tone much
softened, there was a smile upon his face, and he answered freely the
questions put to him about his journey. Overcome at first by the dark
aspect of this home-coming, he now began to taste the joy of having
Clara under his roof, rescued alike from those vague dangers of the past
and from the recent peril. Impossible to separate the sorrow he felt for
her blighted life, her broken spirit, and the solace lurking in the
thought that henceforth she could not abandon him. Never a word to
reproach her for the unalterable; it should be as though there were no
gap between the old love and its renewal in the present. For Clara used
to love him, and already she had shown that his tenderness did not
appeal to her in vain; during the journey she had once or twice pressed
his hand in gratitude. How well it {}was that ho had this home in which
to receive her! Half a year ago, and what should he have done? He would
not admit to himself that there were any difficulties ahead; if it came
to that, he would manage to get some extra work in the evening and on
Saturday afternoons. He would take Sidney into council. But thereupon
his face darkened again, and he lost himself in troubled musing.

Midway in the Sunday morning Amy told him that Clara had risen and would
like him to go and sit with her. She would not leave her room; Amy had
put it in order, and the blind was drawn low. Clara sat by the fireside,
in her attitude of last night, hiding her face as far as she was able.
The beauty of her form would have impressed any one who approached her,
the grace of her bent head; but the countenance was no longer that of
Clara Hewett; none must now look at her, unless to pity. Feeling herself
thus utterly changed, she could not speak in her former natural voice;
her utterance was oppressed, unmusical, monotonous.

When her father had taken a place near her {}she asked him, ``Have you
got that piece of newspaper still?''

He had, and at her wish produced it. Clara held it in the light of the
fire, and regarded the pencilled words closely. Then she inquired if he
wished to keep it, and on his answering in the negative threw it to be
burnt. Hewett took her hand, and for a while they kept silence.

``Do you live comfortably here, father?'' she said presently.

``We do, Clara. It's a bit high up, but that don't matter much.''

``You've got new furniture,''

``Yes, some new things. The old was all done for, you know.''

``And where did you live before you came here?''

``Oh, we had a place in King's Cross Road,---it wasn't much of a place,
but I suppose it might a' been worse.''

``And that was where?''{{------}}

``Yes---yes---it was there.''

``And how did you manage to buy this furniture? '' Clara asked, after a
pause.

{}``Well, my dear, to tell you the truth---it was a friend as--- an old
friend helped us a bit.''

``You wouldn't care to say who it was?''

John was gravely embarrassed. Clara moved her head a little, so as to
regard him, but at once turned away, shrinkingly, when she met his eyes.

``Why don't you like to tell me, father? Was it Mr. Kirkwood?''

``Yes, my dear, it was.''

Neither spoke for a long time. Clara's head sank lower; she drew her
hand away from her father's, and used it to shield her face! When she
spoke, it was as if to herself.

``I suppose he's altered in some ways?''

``Not much; I don't see much change, myself, but then of course . No,
he's pretty much the same.''

``He's married, isn't he?''

``Married? Why, what made you think that, Clara? No, not he. He had to
move not long ago; his lodgin's is in Red Lion Street now.''

``And does he ever come here?''

{}``He has been,---just now an' then.''

``Have you told him?''

``Why--- yes, clear,--- I felt I had to.''

``There's no harm. You couldn't keep it a secret. But he mustn't come
whilst I'm here; you understand that, father? ''

``No, no, he shan't. He shall never come, if you don't wish it.''

``Only whilst I'm here.''

``But---Clara---you'll \emph{always} be here.''

``Oh no! Do you think I'm going to burden you all the rest of my life? I
shall find some way of earning a living, and then I shall go and get a
room for myself''

``Now don't--- now don't talk like that!'' exclaimed her father, putting
his hand on her.

``You shall do what else you like, my girl, but don't talk about goin'
away from me. That's the one thing as I couldn't bear. I ain't so young
as I was, and I've had things as was hard to go through,---I mean when
the mother died and---and other things at that time. Let you an' me stay
by each other whilst we may, my girl. You know it was {}always you as I
thought most of, and I want to keep you by me---I do, Clara. You won't
speak about goin' away?''

She remained mute. Shadows from the firelight rose and fell upon the
walls of the half-darkened room. It was a cloudy morning; every now and
then a gust flung rain against the window.

``If you went,'' he continued, huskily, ``I should be afraid o' myself.
I haven't told you. I didn't behave as I'd ought to have done to the
poor mother, Clara; I got into drinkin' too much; yes, I did. I've broke
myself off that; but if you was to leave me{{------}} I've had hard
things to go through. Do you know the Burial Club broke up just before
she died? I couldn't get not a ha'penny! A lot o' the money was stolen.
You may think how I felt, Clara, with her lyin' there, and I hadn't got
as much as would pay for a coffin. It was Sidney Kirkwood found the
money,---he did! There was never man had as good a friend as he's been
to me; I shall never have a chance of pay in' what I owe {}him. Things
is better with me now, but I'd rather beg my bread in the streets than
you should go away. Don't be afraid, my dearest. I promise you nobody
shan't come near. You won't mind Mrs. Eagles; she's very good to the
children. But I must keep you near to me, my poor girl!''

Perhaps it was that word of pity,---though the man's shaken voice was
throughout deeply moving. For the first time since the exultant hope of
her life was blasted, Clara shed tears.

~

END OF VOL II.

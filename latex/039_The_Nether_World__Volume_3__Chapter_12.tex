\chapter{Sidney}

\textsc{Look} at a map of greater London, a map on which the town proper
shows as a dark, irregularly rounded patch against the whiteness of
suburban districts, and just on the northern limit of the vast network
of streets you will distinguish the name of Crouch End. Another decade,
and the dark patch will have spread greatly further; for the present.
Crouch End is still able to remind one that it was in the country a very
short time ago. The streets have a smell of newness, of dampness; the
bricks retain their complexion, the stucco has not rotted more than one
expects in a year or two; poverty tries to hide itself with Venetian
blinds, until the time when an advanced guard of houses shall justify
the existence of the slum.

Characteristic of the locality is a certain row of one-storey
cottages,---villas, the {}advertiser calls them,---built of white brick,
each with one bay window on the ground floor, a window pretentiously
fashioned and desiring to be taken for stone, though obviously made of
bad plaster. Before each house is a garden, measuring six feet by three,
entered by a little iron gate, which grinds as you push it, and at no
time would latch. The front-door also grinds on the sill; it can only be
opened by force, and quivers in a way that shows how unsubstantially it
is made. As you set foot in the pinched passage, the sound of your tread
proves the whole fabric a thing of lath and sand. The ceilings, the
walls, confess themselves neither water-tight nor air-tight. Whatever
you touch is at once found to be sham.

In the kitchen of one of these houses, at two o'clock on a Saturday
afternoon in September, three young people were sitting down to the
dinner-table: a girl of nearly fourteen, her sister, a year younger, and
their brother, not yet eleven. All were decently dressed, but very
poorly; a glance at them, and you knew that in this house there was
little money to spend on superfluities. The same impression {}was
produced by the appointments of the kitchen, which was disorderly, too,
and spoke neglect of the scrubbing-brush. As for the table, it was ill
laid and worse supplied. The meal was to consist of the fag-end of a
shoulder of mutton, some villainously cooked potatoes (\emph{à
l'Anglaise}) and bread.

``Oh, I can't eat this rot again!'' cried the boy, making a dig with his
fork at the scarcely clad piece of bone. ``I shall have bread and
cheese. Lug the cheese out, Annie!''

``No, you won't,'' replied the elder girl, in a disagreeable voice.
``You'll eat this or go without.''

She had an unpleasing appearance. Her face was very thin, her lips
pinched sourly together, her eyes furtive, hungry, malevolent. Her
movements were awkward and impatient, and a morbid nervousness kept her
constantly starting, with a stealthy look here or there.

``I shall have the cheese if I like!'' shouted the boy, a very
ill-conditioned youngster, whose face seemed to have been damaged in
recent conflict. His clothes were dusty, and his hair stood up like
stubble.

``Hold your row, Tom,'' said the younger {}girl, who was quiet, and had
the look of an invalid. ``It's always you begins. Besides, you can't
have cheese; there's only a little bit, and Sidney said he was going to
make his dinner of it to-day.''

``Of course,---selfish beast!''

``Selfish! Now just listen to that. Amy! when he said it just that we
mightn't be afraid to finish the meat.''

Amy said nothing, but began to hack fragments off the bone.

``Put some aside for father first,'' continued Annie, holding a plate.

``Father be blowed!'' cried Tom. ``You just give me that first cut. Give
it here, Annie, or I'll crack you on the head!''

As he struggled for the plate, Amy bent forward and hit his arm
violently with the handle of the knife. This was the signal for a
general scrimmage, in the midst of which Tom caught up a hearth-brush
and flung it at Amy's head. The missile went wide of its mark and
shivered one of the window-panes.

``There now!'' exclaimed Annie, who had begun to cry in consequence of a
blow from Tom's fist. ``See what father says to that!''

{}``If I was him,'' said Amy, in a low voice of passion, ``I'd tie you
to something and beat you till you lost your senses. Ugly brute!''

The warfare would not have ended here but that the door opened and he of
whom they spoke made his appearance.

In the past two years and a half John Hewett had become a shaky old man.
Of his grizzled hair very little remained, and little of his beard; his
features were shrunken, his neck scraggy; he stooped much, and there was
a senile indecision in his movements. He wore rough, patched clothing,
had no collar, and seemed, from the state of his hands, to have been
engaged in very dirty work. As he entered and came upon the riotous
group his eyes lit up with anger. In a strained voice he shouted a
command of silence.

``It's all that Tom, father,'' piped Annie. ``There's no living with
him.''

John's eye fell on the broken window.

``Which of you's done that?'' he asked sternly, pointing to it.

No one spoke.

``Who's goin' to pay for it, I'd like to know? Doesn't it cost enough to
keep you, {}but you must go makin' extra expense? Where's the money to
come from, I want to know, if you go on like this?''

He turned suddenly upon the elder girl.

``I've got something to say to you, Miss. Why wasn't you at work this
morning?''

Amy avoided his look. Her pale face became mottled with alarm, but only
for an instant; then she hardened herself and moved her head insolently.

``Why wasn't you at work? Where's your week's money?''

``I haven't got any.''

``You haven't got any? Why not?''

For a while she was stubbornly silent, but Hewett constrained her to
confession at length. On his way home to-day he had been informed by an
acquaintance that Amy was wandering about the streets at an hour when
she ought to have been at her employment. Unable to put off the evil
moment any longer, the girl admitted that four days ago she was
dismissed for bad behaviour, and that since then she had pretended to go
to work as usual. The trifling sum paid to her on dismissal she had
spent.

{}Jolin turned to his youngest daughter and asked in a hollow voice:

``Where's Clara?''

``She's got one of her headaches, father,'' replied the girl, trembling.

He turned and went from the room.

It was long since he had lost his place of porter at the filter-works.
Before leaving England, Joseph Snowdon managed to dispose of his
interest in the firm of Lake, Snowdon, \& Co., and at the same time
Hewett was informed that his wages would be reduced by five shillings a
week,---the sum which had been supplied by Michael Snowdon's
benevolence. It was a serious loss. Clara's marriage removed one grave
anxiety, but the three children had still to be brought up, and with
every year John's chance of steady employment would grow less. Sidney
Kirkwood declared himself able and willing to help substantially, but he
might before long have children of his own to think of, and in any case
it was shameful to burden him in this way.

Shameful or not, it very soon came to pass that Sidney had the whole
family on his hands. {}A bad attack of rheumatism in the succeeding
winter made John incapable of earning anything at all; for two months he
was a cripple. Till then Sidney and his wife had occupied lodgings in
Holloway; when it became evident that Hewett must not hope to be able to
support his children, and when Sidney had for many weeks paid the rent
(as well as supplying the money to live npon) in Farringdon Road
Buildings, the house at Crouch End was taken, and there all went to live
together. Clara's health was very uncertain, and though at first she
spoke frequently of finding work to do at home, the birth of a child put
an end to such projects. Amy Hewett was shortly at the point when the
education of a board-school child is said to be ``finished;'' by good
luck, employment was found for her in Kentish Town, with three shillings
a week from the first. John could not resign himself to being a mere
burden on the home. Enforced idleness so fretted him that at times he
seemed all but out of his wits. In despair he caught at the strangest
kinds of casual occupation; when earning nothing, he would barely eat
enough to keep himself alive, and if he succeeded in bringing {}home a
shilling or two, he turned the money about in his hands with a sort of
angry joy that it would have made your heart ache to witness. Just at
present he had a job of cleaning and whitewashing some cellars in Stoke
Newington.

He was absent from the kitchen for five minutes, during which time the
three sat round the table. Amy pretended to eat unconcernedly; Tom made
grimaces at her. As for Annie, she cried. Their father entered the room
again.

``Why didn't you tell us about this at once?'' he asked, in a shaking
voice, looking at his daughter with eyes of blank misery.

``I don't know.''

``You're a bad, selfish girl!'' he broke out, again overcome with anger.
``Haven't you got neither sense nor feelin' nor honesty? Just when you
ought to have begun to earn a bit higher wages,---when you ought to have
been glad to work your hardest, to show you wasn't unthankful to them as
has done so much for you! Who earned money to keep you when you was
goin' to school? Who fed and clothed you, and saw as you didn't want
{}for nothing? Who is it as you owe everything to?---just tell me
that.''

Amy affected to pay no attention. She kept swallowing morsels, with ugly
movements of her lips and jaws.

``How often have I to tell you all that if it wasn't for Sidney Kirkwood
you'd have been workhouse children? As sure as you're livin', you'd all
of you have gone to the workhouse! And you go on just as if you didn't
owe thanks to nobody. I tell you it'll be years and years before one of
you'll have a penny you can call your own. If it was Annie or Tom
behaved so careless, there'd be less wonder; but for a girl of your
age---I'm ashamed as you belong to me! You can't even keep your tongue
from bein' impudent to Clara, her as you ain't worthy to be a servant
to!''

``Clara's a sneak,'' observed Tom, with much coolness. ``She's always
telling lies about us.''

``I'll half-knock your young head off your shoulders,'' cried his
father, furiously, ``if you talk to me like that! Not one of you's fit
to live in the same house with her.''

``Father, I haven't done nothing,'' whimpered {}Annie, hurt by being
thus included in his reprobation.

``No more you have---not just now, but you're often enough more trouble
to your sister than you need be. But it's you I'm talkin' to, Amy. You
dare to leave this house again till there's another place found for you!
If you'd any self-respect, you couldn't bear to look Sidney in the face.
Suppose you hadn't such a brother to work for you, what would you do,
eh? Who'd buy your food? Who'd pay the rent of the house you live in?''

A noteworthy difference between children of this standing and such as
pass their years of play-time in homes unshadowed by poverty. For these,
life had no illusions. Of every mouthful that they ate, the price was
known to them. The roof over their heads was there by no grace of
Providence, but solely because such-and-such a sum was paid weekly in
hard cash, when the collector came; let the payment fail, and they knew
perfectly well what the result would be. The children of the upper world
could not even by chance give a thought to the sources whence their
needs are {}supplied; speech on such a subject in their presence would
be held indecent. In John Hewett's position, the indecency, the crime,
would have been to keep silence and pretend that the needs of existence
are ministered to as a matter of course.

His tone and language were pitifully those of feeble age. The emotion
proved too great a strain upon his body, and he had at length to sit
down in a tremulous state, miserable with the consciousness of failing
authority. He would have made but a poor figure now upon Clerkenwell
Green. Even as his frame was shrunken, so had the circle of his
interests contracted; he could no longer speak or think on the subjects
which had fired him through the better part of his life; if he was
driven to try and utter himself on the broad questions of social wrong,
of the people's cause, a senile stammering of incoherencies was the only
result. The fight had ever gone against John Hewett; he was one of those
who are born to be defeated. His failing energies spent themselves in
conflict with his own children; the concerns of a miserable home were
all his mind could now cope with.

{}``Come and sit down to your dinner, father,'' Annie said, when he
became silent.

``Dinner? I want no dinner. I've no stomach for food when it's stolen.
What's Sidney goin' to have when he comes home?''

``He said he'd do with bread and cheese to-day. See, we've cut some meat
for you?''

``You keep that for Sidney, then, and don't one of you dare to say
anything about it. Cut me a bit of bread, Annie.''

She did so. He ate it, standing by the fireplace, drank a glass of
water, and went into the sitting-room. There he sat unoccupied for
nearly an hour, his head at times dropping forward as if he were nearly
asleep; but it was only in abstraction. The morning's work had wearied
him excessively, as such effort always did, but the mental misery he was
suffering made him unconscious of bodily fatigue.

The clinking and grinding of the gate drew his attention; he stood up
and saw his son-in-law, returned from Clerkenwell. When he had heard the
house-door grind and shake and close, he called ``Sidney!''

Sidney looked into the parlour, with a smile.

{}``Come in here a minute; I want to speak to you.''

It was a face that told of many troubles. Sidney might resolutely keep a
bright countenance, but there was no hiding the sallowness of his cheeks
and the lines drawn by ever- wakeful anxiety. The effect of a struggle
with mean necessities is seldom anything but degradation, in look and in
character; but Sidney's temper, and the conditions of his life,
preserved him against that danger. His features, worn into thinness,
seemed to present more distinctly than ever their points of refinement.
You saw that he was habitually a grave and silent man; all the more
attractive his aspect when, as now, he seemed to rest from thought and
give expression to his natural kindliness. In the matter of attire he
was no longer as careful as he used to be; the clothes he wore had done
more than just service, and hung about him unregarded.

``Clara upstairs?'' he asked, when he had noticed Hewitt's look.

``Yes; she's lying down. May's been troublesome all the morning. But it
was something else I meant.''

{}And John began to speak of Amy's ill-doing. He had always in some
degree a sense of shame when he spoke privately with Sidney, always felt
painfully the injustice involved in their relations. At present he could
not look Kirkwood in the face, and his tone was that of a man who abases
himself to make confession of guilt.

Sidney was gravely concerned. It was his habit to deal with the
children's faults good-naturedly, to urge John not to take a sombre view
of their thoughtlessness; but the present instance could not be made
light of. Secretly he had always expected that the girl would be a
source of more serious trouble the older she grew. He sat in silence,
leaning forward, his eyes bent down.

``It's no good whatever \emph{I} say,'' lamented Hewett. ``They don't
heed me. Why must I have children like these? Haven't I always done my
best to teach them to be honest and good-hearted? If I'd spent my life
in the worst ways a man can, they couldn't have turned out more
worthless. Haven't I wished always what was right and good and true?
Haven't I always spoke up for justice in the {}world? Haven't I done
what I could, Sidney, to be helpful to them as fell into misfortune? And
now in my old age I'm only a burden, and the children as come after me
are nothing but a misery to all as have to do with them. If it wasn't
for Clara I feel I couldn't live my time out. She's the one that pays me
back for the love I've given her. All the others,---I can't feel as
they're children of mine at all.''

It was a strange and touching thing that he seemed now-a-days utterly to
have forgotten Clara's past. Invariably he spoke of her as if she had at
all times been his stay and comfort. The name of his son who was dead
never passed his lips, but of Clara he could not speak too long or too
tenderly.

``I can't think what to do,'' Sidney said. ``If I talk to her in a
fault-finding way, she'll only dislike me the more; she feels I've no
business to interfere.''

``You're too soft with them. You spoil them. Why, there's one of them
broken a pane in the kitchen to-day, and they know you'll take it quiet,
like you do everything else.''

Sidney wrinkled his brow. These petty expenses, ever repeated, were just
what made {}the difficulty in his budget; he winced whenever such
demands encroached upon the poor weekly income of which every penny was
too little for the serious needs of the family. Feeling that if he sat
and thought much longer a dark mood would seize upon him, he rose
hastily.

``I shall try kindness with her. Don't say anything more in her
hearing.''

He went to the kitchen-door, and cried cheerfully, ``My dinner ready,
girls?''

Annie's voice replied with a timorous affirmative.

``All right; I'll be down in a minute.''

Treading as gently as possible, he ascended the stairs and entered his
bedroom. The blind was drawn down, but sunlight shone through it and
made a softened glow in the chamber. In a little cot was sitting his
child. May, rather more than a year old; she had toys about her, and was
for the moment contented. Clara lay on the bed, her face turned so that
Sidney could not see it. He spoke to her, and she just moved her arm,
but gave no reply.

``Do you wish to be left alone?'' he asked, in a subdued and troubled
voice.

{}``Yes.''

``Shall I take May downstairs?''

``If you like. Don't speak to me now.''

He remained standing by the bed for a minute, then turned his eyes on
the child, who smiled at him. He could not smile in return, but went
quietly away.

``It's one of her bad days,'' whispered Hewett, who met him at the foot
of the stairs. ``She can't help it, poor girl!''

``No, no.''

Sidney ate what was put before him without giving a thought to it.
``When his eyes wandered round the kitchen the disorder and dirt worried
him, but on that subject he could not speak. His hunger appeased, he
looked steadily at Amy, and said in a kindly tone:

``Father tells me you've had a stroke of bad luck. Amy. We must have a
try at another place, mustn't we? Hollo, there's a window broken! Has
Tom been playing at cricket in the room, eh?''

The girls kept silence.

``Come and let's make out the list for our shopping this afternoon,'' he
continued. ``I'm {}afraid there'll have to be something: the less for
that window, girls; what do \emph{you} say?''

``We'll do without a pudding to-morrow, Sidney,'' suggested Annie.

``Oh come, now! I'm fond of pudding.''

Thus it was always; if he could not direct by kindness, he would never
try to rule by harsh words. Six years ago it was not so easy for him to
be gentle under provocation, and he would then have made a better
disciplinarian in such a home as this. On Amy and Tom all his rare
goodness was thrown away. Never mind; shall one go over to the side of
evil because one despairs of vanquishing it?

The budget, the budget! Always so many things perforce cut out; always
such cruel pressure of things that \emph{could} not be cut out. In the
early days of his marriage he had accustomed himself to a liberality of
expenditure out of proportion to his income; the little store of savings
allowed him to indulge his kindness to Clara and her relatives, and he
kept putting off to the future that strict revision of outlay which his
position of course demanded. The day when he had no longer a choice came
all too soon; with alarm he discovered that his {}savings bad melted
away: the few sovereigns remaining must be sternly guarded for the hour
of stern necessity. How it ground on his sensibilities when he was
compelled to refuse some request from Clara or the girls! His generous
nature suffered pangs of self-contempt as often as there was talk of
economy. To-day, for instance, whilst he was worrying in thought over
Amy's behaviour, and at the same time trying to cut down the Saturday's
purchases in order to pay for the broken window, up comes Tom with the
announcement that he lost his hat this morning, and had to return
bareheaded. Another unforeseen expense! And Sidney was angry with
himself for his impulse of anger against the boy.

Clara never went out to make purchases, seldom indeed left the house for
any reason, unless Sidney persuaded her to walk a short distance with
him after sundown, when she veiled herself closely. Neither Amy nor
Annie could be trusted to do all the shopping, so that Sidney generally
accompanied one or other of them for that purpose on Saturday afternoon.
To-day he asked Amy to go with him, wishing, if possible, to influence
her for good by kind, {}brotherly talk. Whilst she was getting ready he
took John aside into the parlour, to impart a strange piece of news he
bad brought from Clerkenwell.

``Mrs. Peckover has had a narrow escape of being poisoned. She was found
by one of her lodgers all but dead, and last night the police arrested
her daughter on the charge.''

``Mrs. Snowdon?''

``Yes. The mother has accused her. There's a man concerned in the
affair. One of the men showed me a report in to-day's paper; I didn't
buy one, because we shall have it in the Sunday paper to-morrow. Nice
business, eh?''

``That's for the old woman's money, I'll wager!'' exclaimed Hewett, in
an awed voice. ``I can believe it of Clem; if ever there was a downright
bad 'un! Was she living in the Close?''

``Mrs. Snowdon wasn't. Somewhere in Hoxton. No doubt it was for the
money---if the charge is true. We won't speak of it before the
children.''

``Think of that, now! Many's the time I've looked at Clem Peckover and
said to myself, {}`You'll come to no good end, my lady!' She was a
fierce an' bad 'un.''

Sidney nodded, and went off for his walk with Amy{.~.~.~.}

It was a difficult thing to keep any room in the house orderly, and
Sidney, as part of his struggle against the downward tendency in all
about him, against the forces of chaos, often did the work of housemaid
in the parlour; a little laxity in the rules which made this a sacred
corner, and there would have been no spot where he could rest. With some
success, too, he had resisted the habit prevalent in workingclass homes
of prolonging Saturday evening's occupations until the early hours of
Sunday morning. At a little after ten o'clock to-night John Hewett and
the children were in bed; he too, weary in mind and body, would gladly
have gone upstairs, but he lingered from one five minutes to the next,
his heart sinking at the certainty that he would find Clara in sleepless
misery which he had no power to allay.

Round the walls of the parlour were hung his own drawings, which used to
conceal the bareness of his lodging in Tysoe Street. It was {}three
years since he had touched a pencil; the last time having been when he
made holiday with Michael Snowdon and Jane at the farmhouse by Danbury
Hill. The impulse would never come again. It was associated with,
happiness, with hope; and what had his life to do with one or the other?
Could he have effected the change without the necessity of explaining
it, he would gladly have put those drawings out of sight. Whenever, as
now, he consciously regarded them, they plucked painfully at his
heart-strings, and threatened to make him a coward.

None of that! He had his work to do, happiness or no happiness, and by
all the virtue of manhood he would not fail in it---as far as success or
failure was a question of his own resolve.

The few books he owned were placed on hanging shelves; among them those
which he had purchased for Clara since their marriage. But reading was
as much a thing of the past as drawing. Never a moment when his mind was
sufficiently at ease to refresh itself with other men's thoughts or
fancies. As with John Hewett, so with himself; the circle of {}his
interests had shrivelled, until it included nothing but the cares of his
family, the cost of house and food and firing. As a younger man, he had
believed that he knew what was meant by the struggle for existence in
the nether world; it seemed to him now as if such knowledge had been
only theoretical. Ob, it was easy to preach a high ideal of existence
for the poor, as long as one had a considerable margin over the week's
expenses; easy to rebuke the men and women who tried to forget
themselves in beer-shops and gin-houses, as long as one could take up
some rational amusement with a quiet heart. Now, on his return home from
labour, it was all he could do not to sink in exhaustion and defeat of
spirit. Shillings and pence; shillings and pence;---never a question of
pounds, unfortunately; and always too few of them. He understood how men
have gone mad under pressure of household cares; he realised the
horrible temptation which has made men turn dastardly from the path
leadinor homeward and leave those there to shift for themselves.

When on the point of lowering the lamp he heard some one coming
downstairs. The {}door opened, and, to his surprise, Clara came in.
Familiarity could not make him insensible to that disfigurement of her
once beautiful face; his eyes always fell before her at the first moment
of meeting;.

``What are you doing?'' she asked. ``Why don't you come up?''

``I was that minute coming.''

His hand went again to the lamp, but she checked him. In a low, wailing,
heart-breaking voice, and with a passionate gesture, she exclaimed,
``Oh, I feel as if I should go mad! I can't bear it much longer!''

Sidney was silent at first, then said quietly, ``Let's sit here for a
little. No wonder you feel low-spirited, lying in that room all day. I'd
gladly have come and sat with you, but my company only seems to irritate
you.''

``What good can you do me 1 You only think I'm making you miserable
without a cause. You won't say it, but that's what you always think; and
when I feel that, I can't bear to have you near. If only I could die and
come to the end of it! How can you tell what I suffer? Oh yes, you speak
so calmly---as good as telling me I am {}unreasonable because I can't do
the same. I hate to hear your voice when it's like that! I'd rather you
ragged at me or struck me!''

The beauty of her form had lost nothing since the evening when he
visited her in Farringdon Road Buildings; now, as then, all her
movements were full of grace and natural dignity. Whenever strong
feeling was active in her, she could not but manifest it in motion
unlike that of ordinary women. Her hair hung in disorder, though not at
its full length, massing itself upon her shoulders, shadowing her
forehead. Half-consumed by the fire that only death would extinguish,
she looked the taller for her slenderness. Ah, had the face been
untouched!

``You are unjust to me,'' Sidney replied, with emotion, but not
resentfully. ``I can enter into all your sufferings. If I speak calmly,
it's because I \emph{must}, because I daren't give way. One of us must
try and be strong, Clara, or else{{------}}''

He turned away.

``Let us leave this house,'' she continued, hardly noticing what he
said. ``Let us live in some other place. Never any {}change---always,
always the same walls to look at, day and night---it's driving me mad!''

``Clara, we can't move. I daren't spend even the little money it would
cost. Do you know what Amy has been doing?''

``Yes; father told me.''

``How can we go to the least needless expense, when every day makes
living harder for us?''

``What have we to do with them? How can you be expected to keep a whole
family? It isn't fair to you or to me. You sacrifice me to them. It's
nothing to you what I endure, so long as they are kept in comfort!''

He stepped nearer to her.

``What do you really mean by that? Is it seriously your wish that I
should tell them---your father and your sisters and your brother---to
leave the house and support themselves as best they can? Pray, what
would become of them? Kept in \emph{comfort}, are they? How much comfort
does your poor father enjoy? Do you wish me to tell him to go out into
the street, as I can help him no more?''

She moaned and made a wild gesture.

``You know all this to be impossible; you {}don't wish it; you couldn't
bear it. Then why will you drive me almost to despair by complaining so
of what can't be helped? Surely you foresaw it all. You knew that I was
only a working man. It isn't as if there had been any hope of my making
a larger income, and you were disappointed.''

``Does it make it easier to bear because there is no hope of relief?''
she cried.

``For me, yes. If there \emph{were} hope, I might fret under the
misery.''

``Oh, I had hope once! It might have been so different with me. The
thought burns and burns and burns, till I am frantic. You don't help me
to bear it. You leave me alone when I most need help. How can you know
what it means to me to look back and think of what might have been? You
say to yourself I am selfish, that I ought to be thankful some one took
pity on me, poor, wretched creature that I am. It would have been kinder
never to have come near me. I should have killed myself long ago, and
there an end. You thought it was a great thing to take me, when you
might have had a wife who would''{{------}}

``Clara! Clara! When you speak like that, {}I could almost believe you
are really mad. For Heaven's sake, think what you are saying! Suppose I
were to reproach you with having consented to marry me? I would rather
die than let such a word pass my lips,---but suppose you heard me
speaking to you like this?''

She drew a deep sigh, and let her hands fall. Sidney continued in quite
another voice:

``It's one of the hardest things I have to bear, that I can't make your
life pleasanter. Of course you need change; I know it only too well. You
and I ought to have our holiday at this time of the year, like other
people. I fancy I should like to go into the country myself; Clerkenwell
isn't such a beautiful place that one can be content to go there day
after day, year after year, without variety. But we have no money.
Suffer as we may, there's no help for it---because we have no money.
Lives may be wasted---worse, far worse than wasted---just because there
is no money. At this moment a whole world of men and women is in pain
and sorrow---because they have no money. How often have we said that?
The world is made so; everything has to be bought with money.''

{}``You find it easier to bear than I do.''

``Yes; I find it easier. I am stronger-bodied, and at all events I have
some variety, whilst you have none. I know it. If I could take your
share of the burden, how gladly I'd do so! If I could take your
suffering upon myself, you shouldn't be unhappy for another minute. But
that is another impossible thing. People who are fortunate in life may
ask each day what they \emph{can} do; we have always to remind ourselves
what we \emph{can't}."

``You take a pleasure in repeating such things; it shows how little you
feel them.''

``It shows how I have taken to heart the truth of them.''

She waved her hand impatiently, again sighed, and moved towards the
door.

``Don't go just yet,'' said Sidney. ``We have more to say to each
other.''

``I have nothing more to say. I am miserable, and you can't help me.''

``I can, Clara.''

She looked at him with wondering, estranged eyes. ``How? What are you
going to do?''

``Only speak to you, that's all. I have nothing to give but words.
But''{{------}}

{}She would have left him. Sidney stepped forward and prevented her.

``No; you \emph{must} hear what I have got to say. They may be only
words, but if I have no power to move you with my words, then our life
has come to utter ruin, and I don't know what dreadful things lie before
us.''

``I can say the same,'' she replied, in a despairing tone.

``But neither you nor I shall say it! As long as I have strength to
speak, I won't consent to say that! Clara, you must put your hand in
mine, and think of your life and mine as one. If not for my sake, then
for your child's. Think; do you wish May to suffer for the faults of her
parents?''

``I wish she had never been born!''

``And yet you were the happier for her birth. It's only these last six
months that you have fallen again into misery. You indulge it, and it
grows worse, harder to resist. You may say that life seems to grow
worse. Perhaps so. This affair of Amy's has been a heavy blow, and we
shall miss the little money she brought; goodness knows when another
place will be found for her. But all the more {}reason why we should
help each other to struggle. Perhaps just this year or two will be our
hardest time. If Amy and Annie and Tom were once all earning something,
the worst would be over---wouldn't it? And can't we find strength to
hold out a little longer, just to give the children a start in life,
just to make your father's last years a bit happier? If we manage it,
shan't we feel glad in looking back? Won't it be something worth having
lived for?''

He paused, but Clara had no word for him.

``There's Amy. She's a hard girl to manage, partly because she has very
bad health. I always think of that---or try to---when she irritates me.
This afternoon I took her out with me, and spoke as kindly as I could;
if she isn't better for it, she surely can't be worse, and in any case I
don't know what else to do. Look, Clara, you and I are going to do what
we can for these children; we're not going: to give up the work now
we've begun it. Mustn't all of us who are poor stand together and help
one another? We have to fight against the rich world that's always
crushing us down, down---whether it means to or not. Those people enjoy
their lives. Well, I shall find {}\emph{my} enjoyment in defying them to
make me despair! But I can't do without your help. I didn't feel very
cheerful as I sat here a while ago, before you came down; I was almost
afraid to go upstairs, lest the sight of what you were suffering should
be too much for me. Am I to ask a kindness of you and be refused,
Clara?''

It was not the first time that she had experienced the constraining
power of his words when he was moved with passionate earnestness. Her
desire to escape was due to a fear of yielding, of suffering her egotism
to fail before a stronger will.

``Let me go,'' she said, whilst he held her arm. ``I feel too ill to
talk longer.''

``Only one word---only one promise---now whilst we are the only ones
awake in the house. We are husband and wife, Clara, and we must be kind
to each other. We are not going to be like the poor creatures who let
their misery degrade them. We are both too proud for that---what? We can
think and express our thoughts; we can speak to each other's minds and
hearts. Don't let us be beaten!''

``What's the good of my promising? I can't keep it. I suffer too much.''

{}``Promise, and keep the promise for a few weeks, a few days; then I'll
find strength to help you once more. But now it's your turn to help me.
To-morrow begins a new week; the rich world allows us to rest to-morrow,
to be with each other. Shall we make it a quiet, restful, hopeful day?
When they go out in the morning, you shall read to father and me---read
as you know how to, so much better than I can. What? Was that really a
smile?''

``Let me go, Sidney. Oh, I'm tired, I'm tired!''

``And the promise?''

``I'll do my best. It won't last long, but I'll try.''

``Thank you, dear.''

``No,'' she replied, despondently. ``It's I that ought to thank you. But
I never shall,---never. I only understand you now and then---just for an
hour---and all the selfishness comes back again. It'll be the same till
I'm dead.''

He put out the lamp and followed her upstairs. His limbs ached; he could
scarcely drag one leg after the other. Never mind; the battle was gained
once more.

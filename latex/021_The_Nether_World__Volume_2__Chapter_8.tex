\chapter{Death the Reconciler}

\textsc{There} is no accounting for tastes. Sidney Kirkwood, spending
his Sunday evening in a garden away there in the chawbacon regions of
Essex, where it was so deadly quiet that you could hear the flutter of a
bird's wing or the rustle of a leaf, not once only congratulated himself
on his good fortune; yet at that hour he might have stood, as so often,
listening to the eloquence, the wit, the wisdom, that give proud
distinction to the name of Clerkenwell Green. Towards sundown, that
modern Agora rang with the voices of orators, swarmed with listeners,
with disputants, with mockers, with indifferent loungers. The circle
closing about an agnostic lecturer intersected with one gathered for a
prayer-meeting; the roar of an enthusiastic {}total-abstainer blended
with the shriek of a Radical politician. Innumerable were the little
groups which had broken away from the larger ones to hold semi-private
debate on matters which demanded calm consideration and the finer
intellect. From the doctrine of the Trinity to the question of cabbage
\emph{versus} beef; from Neo-Malthusianism to the grievance of
compulsory vaccination; not a subject which modernism has thrown out to
the multitude but here received its sufficient mauling. Above the crowd
floated wreaths of rank tobacco smoke.

Straying from circle to circle might have been seen Mr. Joseph Snowdon,
the baldness of his crown hidden by a most respectable silk hat, on one
hand a glove, in the other his walking-stick, a yellow waistcoat
enhancing his appearance of dignity, a white necktie spotted with blue
and a geranium in his buttonhole correcting the suspicion of age
suggested by his countenance. As a listener to harangues of the most
various tendency, Mr. Snowdon exhibited an impartial spirit; {}he smiled
occasionally, but was never moved to any expression of stronger feeling.
His placid front revealed the philosopher.

Yet at length something stirred him to a more pronounced interest. He
was on the edge of a dense throng which had just been delighted by the
rhetoric of a well-known Clerkenwell Radical; the topic under discussion
was Kent, and the last speaker had, in truth, put before them certain
noteworthy views of the subject as it affected the poor of London. What
attracted Mr. Snowdon's attention was the voice of the speaker who next
rose. Pressing a little nearer, he got a glimpse of a lean, haggard,
grey-headed man, shabbily dressed, no bad example of a sufferer from the
hardships he was beginning to denounce. ``That's old Hewett,'' remarked
somebody close by. ``He's the feller to let 'em 'ave it!'' Yes, it was
John Hewett, much older, much more broken, yet much fiercer than when we
last saw him. Though it was evident that he spoke often at these
meetings, he had no command of his voice {}and no coherence of style;
after the first few words he seemed to be overcome by rage that was
little short of frenzy. Inarticulate screams and yells interrupted the
torrent of his invective; he raised both hands above his head and
clenched them in a gesture of frantic passion; his visage was
frightfully distorted, and in a few minutes there actually fell drops of
blood from his bitten lip. Rent!---it was a subject on which the poor
fellow could speak to some purpose. What was the root of the difficulty
a London workman found in making both ends meet? Wasn't it that accursed
law by which the owner of property can make him pay a half, and often
more, of his earnings for permission to put his wife and children under
a roof? And what sort of dwellings were they, these in which the men who
made the wealth of the countiy were born and lived and died? What would
happen to the landlords of Clerkenwell if they got their due? Ay, what
\emph{shall} happen, my boys, and that before so very long? For fifteen
or twenty minutes John expended his {}fury, until, in fact, he was
speechless. It was terrible to look at him when at length he made his
way out of the crowd; his face was livid, his eyes bloodshot, a. red
slaver covered his lips and beard; you might have taken him for a
drunken man, so feebly did his limbs support him, so shattered was he by
the fit through which he had passed.

Joseph followed him, and presently walked along at his side.

``That was about as good a speech as I've heard for a long time, Mr.
Hewett,'' he began by observing. ``I like to hear a man speak as if he
meant it.''

John looked up with a leaden, rheumy eye, but the compliment pleased
him, and in a moment he smiled vacantly.

``I haven't said my last word yet,'' he replied, with difficulty making
himself audible through his hoarseness.

``It takes it out of you, I'm afraid. Suppose we have a drop of
something at the corner here?''

``I don't mind, Mr. Snowdon. I thought {}of looking in at my club for a
quarter of an hour; perhaps you'd come round with me afterwards?''

They drank at the public-house, then Hewett led the way by back streets
to the quarters of the club of which he had been for many years a
member. The locality was not cheerful, and the house itself stood in
much need of repair. As they entered, John requested his companion to
sign his name in the visitors' book; Mr. Snowdon did so with a flourish.
They ascended to the first floor and passed into a room where little
could be seen but the gas-jets, and those dimly, owing to the fume of
pipes. The rattle of bones, the strumming of a banjo, and a voice raised
at intervals in a kind of whoop announced that a nigger entertainment
was in progress. Recreation of this kind is not uncommon on Sunday
evening at the workmen's clubs; you will find it announced in the
remarkable list of lectures, \&c., printed in certain Sunday newspapers.
The company which was exerting itself in the present instance had at all
{}events an appreciative audience; laughter and applause broke forth
very frequently.

``I'd forgot it was this kind o' thing tonight,'' said Hewett, when he
could discover no vacant seat. ``Do you care about it? No more don't I;
let's go down into the readin'room.''

Downstairs they established themselves at their ease. John ordered two
half-pints of ale,---the club supplied refreshment for the body as well
as for the mind,---and presently he was more himself.

``How's your wife?'' inquired Joseph.

``Better, I hope?''

``I wish I could say so,'' answered the other, shaking his head. ``She
hasn't been up since Thursday. She's bad, poor woman! she's bad.''

Joseph murmured his sympathy, between two draughts of ale.

``Seen young Kirkwood lately?'' Hewett asked, averting his eyes and
assuming a tone of half-absent indifference.

``He's gone away for his holiday; gone into {}Essex somewhere. When was
it he was speaking of you? Why, one day last week, to be sure.''

``Speakin' about me, eh?'' said John, turning his glass round and round
on the table. And as the other remained silent, he added, ``You can tell
him, if you like, that my wife's been very bad for a long time. Him an'
me don't have nothing to say to each other,---but you can tell him that,
if you like.''

``So I will,'' replied Mr. Snowdon, nodding with a confidential air.

He had noticed from the beginning of his acquaintance with Hewett that
the latter showed no disinclination to receive news of Kirkwood. As
Clem's husband, Joseph was understood to be perfectly aware of the state
of things between the Hewetts and their former friend, and in a recent
conversation with Mrs. Hewett he had assured himself that she, at all
events, would be glad if the estrangement could come to an end. For
reasons of his own, Joseph gave narrow attention to these signs.

{}The talk was turning to other matters, when a man who had just entered
the room and stood looking about him with an uneasy expression caught
sight of Hewett and approached him. He was middle-aged, coarse of
feature, clad in the creased black which a certain type of artisan wears
on Sunday.

``I'd like a word with you, John,'' he said, ``if your friend'll
excuse.''

Hewett rose from the table, and they walked together to an unoccupied
spot.

``Have you heard any talk about the Burial Club?'' inquired the man, in
a low voice of suspicion, knitting his eyebrows.

``Heard anything? No. What?''

``Why, Dick Smales says he can't get the money for his boy, as died last
week.''

``Can't get it? Why not?''

``That's just what I want to know. Some o' the chaps is talkin' about it
upstairs. M'Cosh ain't been seen for four or five days. Somebody had
news as he was ill in bed, and now there's no findin' him. I've got a
notion there's something wrong, my boy.''

{}Hewett's eyes grew large and the muscles of his mouth contracted.

``Where's Jenkins?'' he asked abruptly. ``I suppose he can explain it?''

``No, by God, he can't! He won't say nothing, but he's been runnin'
about all yesterday and to-day, lookin' precious queer.''

Without paying any further attention to Snowdon, John left the room with
his companion, and they went upstairs. Most of the men present were
members of the Burial Club in question, an institution of some fifteen
years' standing and in connection with the club which met here for
social and political purposes; they were in the habit, like John Hewett,
of depositing their coppers weekly, thus insuring themselves or their
relatives for a sum payable at death. The rumour that something was
wrong, that the secretary M'Cosh could not be found, began to create a
disturbance; presently the nigger entertainment came to an end, and the
Burial Club was the sole topic of conversation.

On the morrow it was an ascertained fact {}that one of the catastrophes
which occasionally befall the provident among wage-earners had come to
pass. Investigation showed that for a long time there had been
carelessness and mismanagement of funds, and that fraud had completed
the disaster, M'Cosh was wanted by the police.

To John Hewett the blow was a terrible one. In spite of his poverty, he
had never fallen behind with those weekly payments. The thing he dreaded
supremely was, that his wife or one of the children should die and he be
unable to provide a decent burial. At the death of the last child born
to him the club had of course paid, and the confidence he felt in it for
the future was a sensible support under the many miseries of his life, a
support of which no idea can be formed by one who has never foreseen the
possibility of those dear to him being carried to a pauper's grave. It
was a touching fact that he still kept up the payment for Clara; who
could say but his daughter might yet come back to him to die? To know
that he had lost that one stronghold {}against fate was a stroke that
left him scarcely strength to go about his daily work.

And he could not breathe a word of it to his wife. Oh that bitter curse
of poverty, which puts corrupting poison into the wounds inflicted by
nature, which outrages the spirit's tenderness, which profanes with
unutterable defilement the secret places of the mourning heart! He could
not, durst not, speak a word of this misery to her whose gratitude and
love had resisted every trial, who had shared uncomplainingly all the
evil of his lot, and had borne with supreme patience those added
sufferings of which he had no conception. For she lay on her deathbed.
The doctor told him so on the very day when he learnt that it would be
out of his power to discharge the fitting pieties at her grave. So far
from looking to her for sympathy, it behoved him to keep from her as
much as a suspicion of what had happened.

Their home at this time was a kitchen in King's Cross Road. The eldest
child, Amy, was now between ten and eleven; Annie was {}nine; Tom seven.
These, of course, went to school every day, and were being taught to
appreciate the woefulness of their inheritance. Amy was, on the whole, a
good girl; she could make purchases as well as her mother, and when in
the mood, look carefully after her little brother and sister; but
already she had begun to display restiveness under the hard discipline
to which the domestic poverty subjected her. Once she had played truant
fi'om school, and told falsehoods to the teachers to explain her
absence. It was discovered that she had been tempted by other girls to
go and see the Lord Mayor's show. Annie and Tom threatened to be
troublesome when they got a little older; the boy could not be taught to
speak the truth, and his sister was constantly committing petty thefts
of jam, sugar, even coppers; and during the past year their mother was
seldom able to exert herself in correcting these faults. Only by dint of
struggle which cost her agonies could she discharge the simplest duties
of home. She made a brave fight against disease and penury {}and
incessant dread of the coming day, but month after month her strength
failed. Now at length she tried vainly to leave her bed. The last
reserve of energy was exhausted, and the end near.

After her death, what then? Through the nights of this week after her
doom had been spoken she lay questioning the future. She knew that but
for her unremitting efforts Hewett would have yielded to the despair of
a drunkard; the crucial moment was when he found himself forsaken by his
daughter, and no one but this poor woman could know what force of loving
will, what entreaties, what tears, had drawn him back a little way from
the edge of the gulf. Throughout his life until that day of Clara's
disappearance he had seemed in no danger from the deadliest enemy of the
poor; one taste of the oblivion that could be bought at any
street-corner, and it was as though drinking had been a recognised habit
with him. A year, two years, and he still drank himself into
forgetfulness as often as his mental suffering waxed {}unendurable. On
the morrow of every such crime---interpret the word rightly---he hated
himself for his cruelty to that pale sufferer whose reproaches were only
the utterances of love. The third year saw an improvement, whether owing
to conscious self-control or to the fact that time was blunting his
affliction. Instead of the public-house, he frequented all places where
the woes of the nether world found fierce expression. He became a
constant speaker at the meetings on Clerkenwell Green and at the Radical
clubs. The effect upon him of this excitement was evil enough, yet not
so evil as the malady of drink. Mrs. Hewett was thankful for the
alternative. But when she was no longer at his side---what then?

His employment was irregular, but for the most part at cabinet-making.
The workshop where he was generally to be found was owned by two
brothers, who invariably spent the first half of each week in steady
drinking. Their money gone, they set to work and made articles of
furniture, which on Saturday they took round to the shops of small
dealers {}and sold for what they could get. When once they took up their
tools, these men worked with incredible persistency, and they expected
the same exertion from those they employed. ``I wouldn't give {a
{{------}}} for the chap as can't do his six- and- thirty hours at the
bench!'' remarked one of them on the occasion of a workman falling into
a fainting-fit, caused by utter exhaustion. Hewett was anything but
strong, and he earned little.

~

Late on Saturday afternoon, Sidney Kirkwood and his friends were back in
London. As he drew near to Tysoe Street, carrying the bag which was all
the luggage he had needed, Sidney by chance encountered Joseph Snowdon,
who, after inquiring about his relatives, said that he had just come
from visiting the Hewetts. Mrs. Hewett was very ill indeed; and it was
scarcely to be expected she would live more than a few days.

``You mean that?'' exclaimed Kirkwood, upon whom, after his week of
holiday and of mental experiences which seemed to have {}changed the
face of the world for him, this sudden announcement came with a painful
shock, reviving all the miserable past. ``She is dying?''

``There's no doubt of it.''

And Joseph added his belief that John Hewett would certainly not take it
ill if the other went there before it was too late.

Sidney had no appetite now for the meal he would have purchased on
reaching home. A profound pity for the poor woman who had given him so
many proofs of her affection made his heart heavy almost to tears. The
perplexities of the present vanished in a revival of old tenderness, of
bygone sympathies and sorrows. He could not doubt but that it was his
duty to go to his former friends at a time such as this. Perhaps, if he
had overcome his pride, he might have sooner brought the estrangement to
an end.

He did not know, and had forgotten to ask of Snowdon, the number of the
house in King's Cross Eoad where the Hewetts lived. He could find it,
however, by visiting {}Pennyloaf. Conquering his hesitation, he was on
the point of going forth, when his landlady came up and told him that a
young girl wished to see him. It was Amy Hewett, and her face told him
on what errand she had come.

``Mr. Kirkwood,'' she began, looking up with embarrassment, for he was
all but a stranger to her now, ``mother wants to know if you'd come and
see her. She's very bad; they're afraid {she's{{------}}''}

The word was choked. Amy had been crying, and the tears again rose to
her eyes.

``I was just coming,'' Sidney answered, as he took her hand and pressed
it kindly.

They crossed Wilmington Square and descended by the streets that slope
to Pentonville Prison. The cellar in which John Hewett and his family
were housed was underneath a milk-shop; Amy led the way down stone steps
from the pavement of the street into an area, where more than two people
would have had difficulty in standing together. Sidney saw that the
window which looked upon this space was draped with a {}sheet. By an
open door they entered a passage, then came to the door of the room. Amy
pushed it open, and showed that a lamp gave light within.

To poor homes Sidney Kirkwood was no stranger, but a poorer than this
now disclosed to him he had never seen. The first view of it made him
draw in his breath, as though a pang went through him. Hewett was not
here. The two younger children were sitting upon a mattress, eating
bread. Amy stepped up to the bedside and bent to examine her mother's
face.

``I think she's asleep,'' she whispered, turning round to Sidney.

Sleep, or death? It might well be the latter, for anything Sidney could
determine to the contrary. The face he could not recognise, or only when
he had gazed at it for several minutes. Oh, pitiless world, that pursues
its business and its pleasure, that takes its fill of life from the
rising to the going down of the sun, and within sound of its clamour is
this hiding-place of anguish and desolation!

{}``Mother, here's Mr. Kirkwood.''

Repeated several times, the words at length awoke consciousness. The
dying woman could not move her head from the pillow; her eyes wandered,
but in the end rested upon Sidney. He saw an expression of surprise, of
anxiety, then a smile of deep contentment.

``I knew you'd come. I did so want to see you. Don't go just yet, will
you?''

The lump in his throat hindered Sidney from replying. Hot tears, an
agony in the shedding, began to stream down his cheeks.

``Where's John?'' she continued, trying to look about the room. ``Amy,
where's your father? He'll come soon, Sidney. I want you and him to be
friends again. He knows he'd never ought to a' said what he did. Don't
take on so, Sidney! There'll be Amy to look after the others. She'll be
a good girl. She's promised me. It's John I'm afraid for. If only he can
keep from drink. Will you try and help him, Sidney?''

There was a terrible earnestness of appeal {}in the look she fixed upon
him. Sidney replied that he would hold nothing more sacred than the
charge she gave him.

``It'll be easier for them to live,'' continued the feeble voice. " I've
been ill so long, and there's been so much expense. Amy '11 be earning
something before long.''

``Don't trouble,'' Sidney answered. ``They shall never want as long as I
live---never!''

``Sidney, come a bit nearer. Do you know anything about \emph{her}?"

He shook his head.

``If ever--- if ever she comes back, don't turn away from her---will
you?''

``I would welcome her as I would a sister of my own.''

``There's such hard things in a woman's life. What would a' become of
me, if John hadn't took pity on me! The world's a hard place; I should
be glad to leave it, if it wasn't for them as has to go on in their
trouble. I knew you'd come when I sent Amy. Oh, I feel that easier in my
mind!''

``Why didn't you send long before? No, {}it's my fault. Why didn't I
come? Why didn't I come?''

There was a footstep in the passage, a slow, uncertain step; then the
door moved a little. With blurred vision Sidney saw Hewett enter and
come forward. They grasped each other's hands, without speaking, and
John, as though his strength were at an end, dropped upon the chair by
the bedside. For the last four or five nights he had sat there; if he
got half-an-hour's painful slumber now and then it was the utmost. His
face was like that of some prisoner, whom the long torture of a foul
dungeon has brought to the point of madness. He uttered only a few words
during the half-hour that Sidney still remained in the room. The latter,
when Mrs. Hewett's relapse into unconsciousness made it useless for him
to stay, beckoned Amy to follow him out into the area and put money in
her hand, begging her to get whatever was needed without troubling her
father. He would come again in the morning.

Mrs. Hewett died just before daybreak {}without a pang, as though death
had compassion on her. When Sidney came, about nine o'clock, he found
Amy standing at the door of the milk-shop; the people who kept it had
brought the children up into their room. Hewett still sat by the bed;
seeing Kirkwood, he pointed to the hidden face.

``How am I to bury her?'' he whispered hoarsely. ``Haven't you heard
about it? They've stole the club-money; they've robbed me of it; I
haven't as much as'll pay for her coffin.''

Sidney fancied at first that the man's mind was wandering, but Hewett
took out of his pocket a scrap of newspaper in which the matter was
briefly reported.

``See, it's there, I've known since last Sunday, and I had to keep it
from her. No need to be afraid of speakin' now. They've robbed me, and I
haven't as much as'll pay for her coffin. It's a nice blasted world,
this is, where they won't let you live, and then make you pay if you
don't want to be buried like a dog! She's had nothing but pain and
{}poverty all her life, and now they'll pitch her out of the way in a
parish box. Do you remember what hopes I used to have when we was first
married? See the end of 'em,---look at this underground hole,---look at
this bed as she lays on! Is it my fault? By God, I wonder I haven't
killed myself before this! I've been drove mad, I tell you,---mad! It's
well if I don't do murder yet; every man as I see go by with a good coat
on his back and a face fat with good feeding, it's all I can do to keep
from catchin' his throat an' tearin' the life out of him!''

``Let's talk about the burial,'' interposed Sidney. ``Make your mind at
ease. I've got enough to pay for all that, and you must let me lend you
what you want.''

``Lend me money? You as I haven't spoke to for years?''

``The more fault mine. I ought to have come back again long since; you
wouldn't have refused an old friend that never meant an unkindness to
you.''

``No, it was me as was to blame,'' said the {}other, with choking voice.
``She always told me so, and she always said what was right. But I can't
take it of you, Sidney; I can't! Lend it? An' where am I goin' to get it
from to pay you back? It won't be so long before I lie like she does
there. It's getting too much for me.''

The first tears he had shed rose at this generosity of the man he had so
little claim upon. His passionate grief and the spirit of rebellion,
which grew more frenzied as he grew older, were subdued to a sobbing
gratitude for the kindness which visited him in his need. Nerveless,
voiceless, he fell back again upon the chair and let his head lie by
that of the dead woman.

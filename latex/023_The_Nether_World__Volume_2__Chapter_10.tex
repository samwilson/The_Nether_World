\chapter{On the Eve of Triumph}

\textsc{``I have} got your letter, but it tells me no more than the last
did. Why don't you say plainly what you mean? I suppose it's something
you are ashamed of. You say that there's a chance for me of earning a
large sum of money, and if you are in earnest, I shall be only too glad
to hear how it's to be done. This life is no better than what I used to
lead years ago; I'm no nearer to getting a good part than I was when I
first began acting, and unless I can get money to buy dresses and all
the rest of it, I may go on for ever at this hateful drudgeiy. I shall
take nothing more from you; I say it, and I mean it; but as you tell me
that this chance has nothing to do with yourself, let me know what it
really is. For a large sum of money {}there are few things I wouldn't
do. Of course it's something disgraceful, but you needn't be afraid on
that account; I haven't lost all my pride yet, but I know what I'm
fighting for, and I won't be beaten. Cost what it may, I'll make people
hear of me and talk of me, and I'll pay myself back for all I've gone
tlirough. So write in plain words, or come and see me.{C. V.''}

She wrote at a round table, shaky on its central support, in the parlour
of an indifferent lodging-house; the October afternoon drew towards
dusk; the sky hung low and murky, or, rather, was itself invisible,
veiled by the fume of factory chimneys; a wailing wind rattled the sash
and the door. A newly lighted fire refused to flame cheerfully,
half-smothered in its own smoke, which every now and then was blown
downwards and out into the room. The letter finished,---scribbled
angrily with a bad pen and in pale ink,---she put it into its
envelope,---``C. H. Scawthorne, Esq.''

{}Then a long reverie, such as she always fell into when alone and
unoccupied. The face was older, but not greatly changed from that of the
girl who fought her dread fight with temptation, and lost it, in the
lodging at Islington, who, then as now, brooded over the wild passions
in her heart and defied the world that was her enemy. Still a beautiful
face, its haughty characteristics strengthened, the lips a little more
sensual, a little coarser; still the same stamp of intellect upon the
forehead, the same impatient scorn and misery in her eyes. She asked no
one's pity, but not many women breathed at that moment who knew more of
suffering.

For three weeks she had belonged to a company on tour in the northern
counties. In accordance with the modern custom---so beneficial to actors
and the public---their repertoiy consisted of one play, the famous
melodrama, ``A Secret of the Thames,'' recommended to provincial
audiences by its run of four hundred and thirty-seven nights at a London
theatre. These, to be sure, were {}not the London actors, but
advertisements in local newspapers gave it to be understood that they
``made an ensemble in no respect inferior to that which was so long the
delight of the metropolis.'' Starred on the placards was the name of Mr.
Samuel Peel, renowned in the north of England; his was the company, and
his the main glory in the piece. As leading lady he had the
distinguished Miss Erminia Walcott; her part was a trying one, for she
had to be half-strangled by ruffians and flung---most decorously---over
the parapet of London Bridge. In the long list of subordinate performers
occurred two names with which we are familiar, Miss Grace Danver and
Miss Clara Vale. The present evening would be the third and last in a
certain town of Lancashire, one of those remarkable centres of industry
which pollute heaven and earth, and on that account are spoken of with
somewhat more of pride than stirred the Athenian when he named his
Acropolis. Clara had just risen to stir the fire, compelled to move by
the smoke that was {}annoying her, when, after a tap at the door, there
carae in a young woman of about five-andtwenty, in a plain walking
costume, tall, very slender, pretty, but looking ill. At this moment
there was a slight flush on her cheeks and a brightness in her eyes
which obviously came of some excitement. She paused just after entering
and said in an eager voice, which had a touch of huskiness:

``What do you think? Miss Walcott's taken her hook!''

Clara did not allow herself to be moved at this announcement. For
several days what is called unpleasantness had existed between the
leading lady and the manager; in other words, they had been quarrelling
violently on certain professional matters, and Miss Walcott had
threatened to ruin the tour by withdrawing her invaluable services. The
menace was at last executed, in good earnest, and the cause of Grace
Danver's excitement was that she, as Miss Walcott's understudy, would
to-night, in all probability, be called upon to take the leading part.

{}``I'm glad to hear it,'' Clara replied, veiy soberly.

``You don't look as if you cared much,'' rejoined the other, with a
little irritation.

``What do you want me to do? Am I to scream with joy because the
greatest actress in the world has got her chance at last?''

There was bitterness in the irony. Whatever their friendship in days
gone by, these two were clearly not on the most amiable terms at
present. This was their first engagement in the same company, and it had
needed but a week of association to put a jealousy and ill-feeling
between them which proved fatal to such mutual kindness as they had
previously cherished. Grace, now no less than in her schooldays, was
fond of patronising; as the elder in years and in experience, she
adopted a tone which Clara speedily resented. To heighten the danger of
a conflict between natures essentially incompatible, both were in a
morbid and nervous state, consumed with discontent, sensitive to the
most trifling injury, abandoned to a fierce egoism, which {}the course
of their lives and the circumstances of their profession kept constantly
inflamed. Grace was of acrid and violent temper; when stung with words
such as Clara was only too apt at using, she speedily lost command of
herself and spoke, or even acted, frantically. Except that she had not
Clara's sensibilities, her lot was the harder of the two; for she knew
herself stricken with a malady which would hunt her unsparingly to the
grave. On her story I have no time to dwell; it was full of
wretchedness, which had caused her, about a year ago, to make an attempt
at suicide. A little generosity, and Clara might have helped to soothe
the pains of one so much weaker than herself; but noble feeling was
extinct in the girl, or so nearly extinct that a breath of petty rivalry
could make her base, cruel, remorseless.

``At all events, I \emph{have} got my chance!'' exclaimed Grace, with a
harsh laugh. ``When you get yours, ask me to congratulate you.''

And she swept her skirts out of the room. In a few minutes Clara put a
stamp on her {}letter and went out to the post. Her presence at the
theatre would not be necessary for another two hours, but as the
distance was slight, and nervousness would not let her remain at home,
she walked on to make inquiry concerning Grace's news. Rain had just
begun to fall, and with it descended the smut and grime that darkened
above the houses; the pavement was speedily over-smeared with sticky
mud, and passing vehicles flung splashes in every direction. Odours of
oil and shoddy, and all such things as characterised the town, grew more
pungent under the heavy shower. On reaching the stage-door, Clara found
two or three of her companions just within; the sudden departure of Miss
Walcott had become known to every one, and at this moment Mr. Peel was
holding a council, to which, as the doorkeeper testified. Miss Danver
had been summoned.

The manager decided to make no public announcement of what had happened
before the hour came for drawing up the curtain. {}A scrappy rehearsal
for the benefit of Grace Danver and the two or three other ladies who
were affected by the necessary rearrangement, went on until the last
possible moment, then Mr. Peel presented himself before the drop and
made a little speech. The gallery was full of mill-hands; in the pit was
a sprinkling of people; the circles and boxes presented half-a-dozen
occupants. ``Sudden domestic calamity \ldots{} enforced absence of the
lady who played \ldots{} efficient substitution \ldots{} deep regret,
but confidence in the friendly feeling of audience on this last
evening.''

They growled, but in the end applauded the actor-manager, who had
succeeded in delicately hinting that, after all, the great attraction
was still present in his own person. The play went very much as usual,
but those behind the scenes were not allowed to forget that Mr. Peel was
in a furious temper; the ladies noticed with satisfaction that more than
once he glared ominously at Miss Danver, who naturally could not aid him
to make his ``points'' as Miss Walcott had accustomed {}herself to do.
At his final exit, it was observed that he shrugged his shoulders and
muttered a few oaths.

Clara had her familiar part; it was a poor one from every point of view,
and the imbecility of the words she had to speak affected her to-night
with exceptional irritation. Clara always acted in ill-humour. She
despised her audience for their acceptance of the playwright's claptrap;
she felt that she could do better than any of the actresses entrusted
with the more important characters; her imagination was for ever turning
to powerful scenes in plays she had studied privately, and despair
possessed her at the thought that she would perhaps never have a chance
of putting forth her strength. To-night her mood was one of sullen
carelessness; she did little more than ``walk through'' her part,
feeling a pleasure in thus insulting the house. One scrap of dialogue
she had with Grace, and her eyes answered with a flash of hatred to the
arrogance of the other's regard. At another point she all but missed her
cue, for her thoughts {}were busy with that letter to which she had
replied this afternoon. Mr. Peel looked at her savagely, and she met his
silent rebuke with an air of indifference. After that the manager
appeared to pay peculiar attention to her as often as they were together
before the footlights. It was not the first time that Mr. Peel had
allowed her to see that she was an object of interest to him.

There was an after-piece, but Clara was not engaged in it. When, at the
fall of the curtain on the melodrama, she went to the shabby
dressing-room which she shared with two companions, a message delivered
by the call-boy bade her repair as soon as possible to the manager's
office. What might this mean? She was startled on the instant, but
speedily recovered her self-control; most likely she was to receive a
rating,---let it come! Without unusual hurry, she washed, changed her
dress, and obeyed the summons.

Mr. Peel was still a young man, of tall and robust stature, sanguine,
with much sham refinement in his manner; he prided himself on {}the
civility with which he behaved to all who had business relations with
him, but every now and then the veneer gave an awkward crack, and, as in
his debate with Miss Walcott, the man himself was discovered to be of
coarse grain. His aspect was singular when, on Clara's entrance into the
private room, he laid down his cigarette and scrutinised her. There was
a fiery hue on his visage, and the scowl of his black eyebrows had a
peculiar ugliness. ``Miss Vale,'' he began, after hesitation, ``do you
consider that you played your part this evening with the
conscientiousness that may fairly be expected of you?''

``Perhaps not,'' replied the girl, averting her eyes, and resting her
hand on the table.

``And may I ask \emph{why} not?"

``I didn't feel in the humour. The house saw no difference.''

``Indeed? The house saw no difference? Do you mean to imply that you
always play badly?''

``I mean that the part isn't worth any attention,---even if they were
able to judge.''

{}There was a perfection of insolence in her tone that in itself spoke
strongly for the abilities she could display if occasion offered.

``This is rather an offhand way of treating the subject, madam,'' cried
Mr. Peel. ``If you disparage our audiences, I beg you to observe that it
is much the same thing as telling me that my own successes are
worthless!''

``I intended nothing of the kind.''

``Perhaps not.'' He thrust his hands into his pockets, and looked down
at his boots for an instant. ``So you are discontented with your part?''

``It's only natural that I should be.''

``I presume you think yourself equal to Juliet, or perhaps Lady
Macbeth?''

``I could play either a good deal better than most women do.''

The manager laughed, by no means ill-humouredly.

``I'm sorry I can't bring you out in Shakespeare just at present, Miss
Vale; but---should you think it a condescension to play Laura Denton?''

{}This was Miss Walcott's part, now Grace Danver's. Clara looked at him
with mistrust; her breath did not come quite naturally.

``How long would it take you, do you think,'' pursued the other, ``to
get the words?''

``An hour or two; I all but know them.''

The manager took a few paces this way and that.

``We go on to Bolton tomorrow morning. Could you undertake to be perfect
for the afternoon rehearsal?''

``Yes.''

``Then I'll try you. Here's a copy you can take. I make no terms, you
understand; it's an experiment. We'll have another talk tomorrow.
Good-night.''

She left the room. Near the door stood Grace Danver and another actress,
both of whom were bidden to wait upon the manager before leaving. Clara
passed under the fire of their eyes, but scarcely observed them.

Rain drenched her between the theatre and her lodgings, for she did not
think of putting up an umbrella; she thought indeed of {}nothing; there
was fire and tumult in her brain. On the round table in her sitting-room
supper was made ready, but she did not heed it. Excitement compelled her
to walk incessantly round and round the scanty space of floor. Already
she had begun to rehearse the chief scenes of Laura Denton; she spoke
the words with all appropriate loudness and emphasis; her gestures were
those of the stage, as though an audience sat before her; she seemed to
have grown taller. There came a double knock at the house-door, but it
did not attract her attention; a knock at her own room, and only when
some one entered was she recalled to the present. It was Grace again;
her lodging was elsewhere, and this late visit could have but one
motive.

They stood face to face. The elder woman was so incensed that her lips
moved fruitlessly, like those of a paralytic.

``I suppose you're going to make a scene,'' Clara addressed her.
``Please remember how late it is, and don't let all the house hear
you.''

``You mean to tell me you accepted that {}offer of Peel's---without
saying a word---without as much as telling him that he ought to speak to
me first?''

``Certainly I did. I've waited long enough; I'm not going to beat about
the bush when my chance comes.''

``And you called yourself my friend?''

``I'm nobody's friend but my own in an affair of this kind. If you'd
been in my place you'd have done just the same.''

``I wouldn't! I \emph{couldn't} have been such a mean creature! Every
man and woman in the company'll cry shame on you.''

``Don't deafen me with your nonsense! If you played the part badly, I
suppose some one else must take it. You were only on trial, like I shall
be.''

Grace was livid with fury.

``Played badly! As if we didn't all know how you've managed it! Much it
has to do with good or bad acting! We know how creatures of your kind
get what they want.''

Before the last word was uttered she was seized with a violent fit of
coughing; her {}cheeks flamed, and spots of blood reddened on the
handkerchief she put to her mouth. Half-stifled, she lay back in the
angle of the wall by the door. Clara regarded her with a contemptuous
pity, and when the cough had nearly ceased, said coldly:

``I'm not going to try and match you in insulting language; I dare say
you'd beat me at that. If you take my advice, you'll go hom.e and take
care of yourself; you look ill enough to be in bed. I don't care what
you or any one else thinks of me; what you said just now was a lie, but
it doesn't matter. I've got the part, and I'll take good care that I
keep it. You talk about us being friends; I should have thought you knew
by this time that there's no such thing as friendship or generosity or
feeling for women who have to make their way in the world. You've had
your hard times as well as I, and what's the use of pretending what you
don't believe? You wouldn't give up a chance for me; I'm sure I should
never expect you to. We have to fight, to fight for everything, and the
weak {}get beaten. That's what life has taught me.''

``You're right,'' was the other's reply, given with a strangely sudden
calmness. ``And we'll see who wins.''

Clara gave no thought to the words, nor to the look of deadly enmity
that accompanied them. Alone again, she speedily became absorbed in a
vision of the triumph which she never doubted was near at hand. A long,
long time it seemed since she had sold herself to degradation with this
one hope. You see that she had formulated her philosophy of life since
then; a child of the nether world whom fate had endowed with intellect,
she gave articulate utterance to what is seething in the brains of
thousands who fight and perish in the obscure depths. The bitter bargain
was issuing to her profit at last; she would yet attain that end which
had shone through all her misery---to be known as a successful actress
by those she had abandoned, whose faces were growing dim to her memory,
but of whom, in truth, she still {}thought more than of all the
multitudinous unknown public. A great success during the remainder of
this tour, and she might hope for an engagement in London. Her portraits
would at length be in the windows; some would recognise her.

Yet she was not so pitiless as she boasted. The next morning, when she
met Grace, there came a pain at her heart in seeing the ghastly,
bloodless countenance which refused to turn towards her. Would Grace be
able to act at all at the next town? Yes, one more scene.

They reached Bolton. In the afternoon the rehearsal took place, but the
first representation was not until to-morrow. Clara saw her name
attached to the leading female character on bills rapidly printed and
distributed through the town. She went about in a dream, rather a
delirium. Mr. Peel used his most affable manner to her; his compliments
after the rehearsal were an augury of great things. And the eventful
evening approached.

{}To give herself plenty of time to dress (the costumes needed for the
part were fortunately simple, and Mr. Peel had advanced her money to
make needful purchases) she left her lodgings at half-past six. It was a
fine evening, but very dark in the two or three by-streets along which
she had to pass to reach the theatre. She waited a minute on the
door-step to let a troop of female mill-hands go by; their shoes clanked
on the pavement, and they were singing in chorus, a common habit of
their kind in leaving work. Then she started and walked quickly\ldots{}

Close by the stage-door, which was in a dark, narrow passage, stood a
woman with veiled face, a shawl muffling the upper part of her body.
Since six o'clock she had been waiting about the spot, occasionally
walking to a short distance, but always keeping her face turned towards
the door. One or two persons came up and entered; she observed them, but
held aloof. Another drew near. The woman advanced, and, as she did so,
freed one of her arms from the shawl.

{}``That you, Grace?'' said Clara, almost kindly, for in her victorious
joy she was ready to be at peace with all the world.

The answer was something dashed violently in her face,---something fluid
and fiery,---something that ate into her flesh, that frenzied her with
pain, that drove her shrieking she knew not whither.

Late in the same night, a pointsman, walking along the railway a little
distance out of the town, came upon the body of a woman, train-crushed,
horrible to view. She wore the dress of a lady; a shawl was still partly
wrapped about her, and her hands were gloved. Nothing discoverable upon
her would have helped strangers in the task of identification, and as
for her {face{{------}}.} But a missing woman was already sought by the
police, and when certain persons were taken to view this body, they had
no difficulty in pronouncing it that of Grace Danver.
